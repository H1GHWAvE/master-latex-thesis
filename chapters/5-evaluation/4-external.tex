%START TEXT INPUT
This is my real text! Rest might be copied or not be checked!
%START TEXT INPUT

%
take away:
dex is dangerous executable format
risks to app developers are significant with no clear solutions

zu ART.
dex isnt dead yet, even with art
still buried deep inside those oat files
far easier to reverse engineer embedded dex than do so for oat

art is a far more advanced runtime architecture, brings android closer to ios native level performance
vestiges of dex still remain to haunt performance, dex code is still 32bit
very much still a shifting landscape, internal structures keep on changing, google isnt afraid to break compatibility, llvm integration likely to only increas eand improve
for most users the change is smooth, better performance and power consumption, negligible cost binary size increase, minor limitations on dex obfuscation remain, for optimal performance and obfuscation nothing beats JNI
\cite{andevconDalvikART}
%
\section{External Improvements} \label{section:evaluation-setion}
sis is text
\subsection{Service-managed Accounts}
\subsection{ART}
\subsection{Secure Elements}
Jetzt waere interessant, ob du dies mal mit SE vergleichen kannst bzw. gibt es irgendwelche Exploits auf dieser Seite? Da letztlich ja jeder Hersteller seine eigenen Applets einbringt und auf den Smartcards bzw. SE weit weniger "Koeche die Suppe versalzen" wuerde ich davon ausgehen, dass die Sicherheit hier hoeher ist. Das ist adhoc aber erstmal nur eine Vermutung.



nachdem du dich mit Secure Elements befasst, moechte ich dich auch auf TEEs hinweisen, die ebenfalls (und nach Aussage von Atredis) fuer DRM genutzt werden. Speziell dortiges QSEE hatte jedoch schwerwiegende Luecken.
\url{https://usmile.at/symposium/program/2015/thomas-holmes​}

DAP Verification .... normalerweise muss jede Applet, die auf so ein Secure Element/Smartcard etc. kommt mit ner Signatur unterschrieben sein ...
\url{http://www.win.tue.nl/pinpasjc/docs/Card%20Spec%20v2.1.1%20v0303.pdf}


Waehrend ich Exploits finden konnte, die Dir erw. Zugriff geben, wenn du Applets installieren kannst, u.a.
\url{https://www.cs.ru.nl/E.Poll/papers/cardis08.pdf}
\url{http://www.uclouvain.be/crypto/wissec2009/static/13.pdf}​



new section trusted execution environment
trusttronic letzte conference
samsung knox
--see-gelten eher sicher
