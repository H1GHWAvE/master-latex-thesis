\section{External Support} \label{section:evaluation-external}
%START TEXT INPUT
This is my real text! Rest might be copied or not be checked!
%START TEXT INPUT

\subsection{Service-managed Accounts} \label{subsection:external-service}
%START TEXT INPUT
This is my real text! Rest might be copied or not be checked!
%START TEXT INPUT


\url{https://youtu.be/TNnccRimhsI?t=1636}\newline
check on server what content should be returned or logic on server\newline

kann man einen lagorithus haben um rauszufinden was man auslagern kann?\newline

if not possible remote code loading\newline

\url{https://www.youtube.com/watch?v=rSH6dnUTDZo}
was ist dann geschützt? content, servers, time constrained urls, obfuscation by using reflection combined with SE -> makes slow but no static analysis\newline

very very slow, e.g 10kHz so no big calculations possible\newline
250bytes, 200ms \newline

\url{http://amies-2014.international-symposium.org/proceedings_2014/Kannengiesser_Baumgarten_Song_AmiEs_2014_Paper.pdf}\newline

\subsection{ART}\label{subsection:external-art}
art hat masschinen coed\newline
wenn reengineerbar dann nicht gut

\subsection{Secure Elements}\label{subsection:counter-external-secure}
%START TEXT INPUT
This is my real text! Rest might be copied or not be checked!
%START TEXT INPUT


best would be to have a security mechanism lucky patcher cannot access, since until now every android version could be rooted using an exploit thus very likely that it will be possible in the future as well
all data thus can be accessed by the user
in order to achieve secure storage introduce secure elements for critical code like license verification
secure space separated from the devecide and its restrictions and vulnerabilities\cite{kannengießerProposal}


can either be mounted in the sdcard slot or using an adapter for the usb interface
accessed over reads and writes to the filesystem
since it has to be small as a sd card and powered by the host system its hardware capabilities are restrained as well \cite{stSe} with power as low as 25MHz complexe calculations would take too much time

can be used to prevent static analysis
encrypted stgrings from the application can be decrypted by the secure element

\url{https://www.youtube.com/watch?v=rSH6dnUTDZo}


%
take away:
dex is dangerous executable format
risks to app developers are significant with no clear solutions

zu ART.
dex isnt dead yet, even with art
still buried deep inside those oat files
far easier to reverse engineer embedded dex than do so for oat

art is a far more advanced runtime architecture, brings android closer to ios native level performance
vestiges of dex still remain to haunt performance, dex code is still 32bit
very much still a shifting landscape, internal structures keep on changing, google isnt afraid to break compatibility, llvm integration likely to only increas eand improve
for most users the change is smooth, better performance and power consumption, negligible cost binary size increase, minor limitations on dex obfuscation remain, for optimal performance and obfuscation nothing beats JNI
\cite{andevconDalvikART}
%

sis is text
\subsection{Service-managed Accounts}
\subsection{ART}
\subsection{Secure Elements}
Jetzt waere interessant, ob du dies mal mit SE vergleichen kannst bzw. gibt es irgendwelche Exploits auf dieser Seite? Da letztlich ja jeder Hersteller seine eigenen Applets einbringt und auf den Smartcards bzw. SE weit weniger \"Koeche die Suppe versalzen\" wuerde ich davon ausgehen, dass die Sicherheit hier hoeher ist. Das ist adhoc aber erstmal nur eine Vermutung.



nachdem du dich mit Secure Elements befasst, moechte ich dich auch auf TEEs hinweisen, die ebenfalls (und nach Aussage von Atredis) fuer DRM genutzt werden. Speziell dortiges QSEE hatte jedoch schwerwiegende Luecken.
%\url{https://usmile.at/symposium/program/2015/thomas-holmes​}

DAP Verification .... normalerweise muss jede Applet, die auf so ein Secure Element/Smartcard etc. kommt mit ner Signatur unterschrieben sein ...
%\url{http://www.win.tue.nl/pinpasjc/docs/Card%20Spec%20v2.1.1%20v0303.pdf}


Waehrend ich Exploits finden konnte, die Dir erw. Zugriff geben, wenn du Applets installieren kannst, u.a.
%\url{https://www.cs.ru.nl/E.Poll/papers/cardis08.pdf}
%\url{http://www.uclouvain.be/crypto/wissec2009/static/13.pdf}​



new section trusted execution environment
trusttronic letzte conference
samsung knox
--see-gelten eher sicher
