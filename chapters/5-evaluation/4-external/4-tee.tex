\subsection{Trusted Execution Environment}\label{subsection:evaluation-external-tee}
%START TEXT INPUT
This is my real text! Rest might be copied or not be checked!
%START TEXT INPUT
luckypatcher not able to attack because it cannot access the TEE

Jetzt waere interessant, ob du dies mal mit SE vergleichen kannst bzw. gibt es irgendwelche Exploits auf dieser Seite? Da letztlich ja jeder Hersteller seine eigenen Applets einbringt und auf den Smartcards bzw. SE weit weniger \"Koeche die Suppe versalzen\" wuerde ich davon ausgehen, dass die Sicherheit hier hoeher ist. Das ist adhoc aber erstmal nur eine Vermutung.

vor und nachteile diskutieren
exploits

nachdem du dich mit Secure Elements befasst, moechte ich dich auch auf TEEs hinweisen, die ebenfalls (und nach Aussage von Atredis) fuer DRM genutzt werden. Speziell dortiges QSEE hatte jedoch schwerwiegende Luecken.
%\url{https://usmile.at/symposium/program/2015/thomas-holmes​}
