\section{Library Modifications} \label{section:evaluation-modifications}
%START TEXT INPUT
This is my real text! Rest might be copied or not be checked!
%START TEXT INPUT
%START TEXT INPUT
This is my real text! Rest might be copied or not be checked!
%START TEXT INPUT

\subsection{Modify the Library} \label{subsection:counter-modifications-library}
%START TEXT INPUT
This is my real text! Rest might be copied or not be checked!
%START TEXT INPUT

\subsection{Native Implementierung} \label{subsection:counter-modifications-dynamic}
%START TEXT INPUT
This is my real text! Rest might be copied or not be checked!
%START TEXT INPUT


reengineering kann aushebeln


NATIVE
Als ein eigenstaendiges Kapitel koenntest du auch noch untersuchen, wie sich Java-Code und Native-Code am Besten mit einander verflechten lassen, um optimalen Schutz gegen den Lucky Patcher zu gewaehrleisten.


Erste Ideen gab es dazu ja bereits - auch von anderen, wie etwa die Verschluesslung von Inhalten und Dekodierung im native Code unter Verwendung von Secure Elements oder Manipulation von Speicherwerten ueber native Libraries, sodass man die Aenderung im (Java) Smali-Code gar nicht mitbekommt. Letzteres hat Herr Hugenroth in einer Seminararbeit einmal grob skizziert (liegt Dir das vor?). Vielleicht fallen Dir weitere Optionen ein? Auch theoretische Ideen sind willkommen.
