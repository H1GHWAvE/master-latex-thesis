\chapter{Cracking Android Applications with LuckyPatcher}\label{chapter:luckypatcher}
http://lucky-patcher.netbew.com/

\section{What is LuckyPatcher and what is it used for?}\label{section:luckypatcher-explain}
wer hat ihn geschrieben?\newline
auf welcher version basiere ich\newline
su nicht vergessen\newline
was kann er alles\newline
was schauen wir uns an?\newline
install apk from palystore -> have root -> open lucky -> chose mode

similar cracking tools:\newline
or manual: decompile and edit what ever you want \newline

\section{Operation}\label{section:luckypatcher-operation}
wo arbeitet er?\newline
warum dex und nicht odex anschauen?\newline
patterns und patching modes grob erklären (modi von luckypatcher die verschiedene operationen (pattern) auf app anwenden) => vorgehensweise zur

\section{What patterns are there and what do they do?}\label{section:luckypatcher-patterns}
was greift jedes pattern an? wie wird der mechanismus ausgeklingt? was ist das result?

\section{What are Patching Modes are there and what do they do?}\label{section:luckypatcher-modes}
kombination von patterns.\newline
welche modes gibt es? welche patterns benutzen sie?\newline
welche apps getestet und welche results?

\section{Learnings from LuckyPatcher}\label{section:luckypatcher-learnings}
was fällt damit weg?\newline
erklären warum 	(2) 5.1.2 Opaque predicates zb nicht geht, da auf dex ebene einfach austauschbar\newline
simple obfuscation for strings? x -> string (damit name egal)
