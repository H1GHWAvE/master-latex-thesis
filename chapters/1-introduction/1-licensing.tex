\section{Licensing} \label{subsection:introduction-licensing}
Software Licensing is the legally binding agreement between two parties regarding the purchase, installation and use of software according to its terms of use.
It defines the rights of the licensor and the licensee.
On the one hand, the goal is to protect the software creator's \gls{ip} or other features and enable him to commercialize it as a product.
On the other hand it defines the boundaries of usage for the user and prevents him from illicit usage \cite{uncgLicensing}.
\newline
\newline
Software licenses come in different variants.
They range from open source to providing limited features or usage for a limited time to purchase only.
Since using the full feature set of software might be bound to paying a royalty fee, these software is often subject of piracy.
In order to prevent unauthorized use, mechanisms are implemented to enforce the legal agreement.
This includes Digital Right Management solutions which deny access to the software in case of a wrong serial key or unregistered account.
\newline
\newline
The problem is that these mechanisms do not offer absolute security and pirates always try to circumvent them.
This results in an everlasting armsrace between software creators and software thefts \cite{szCopy}.
