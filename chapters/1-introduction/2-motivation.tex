\section{Motivation} \label{subsection:introduction-motivation}
Licensing is present in Android, with a market share of almost 82.8\% in Q2 of 2015,  as wel \cite{androidShare}l.
According to Google, this translates to over 1.4 billion active devices in the last 30 days in September 2015 \cite{androidDevices}.
This giant number of Android devices is powered by Google Play \cite{googlePlay}, Google's marketplace.
It offers different kinds of digital goods, as movies, music or ebooks, but also hardware.
In the application section of Google Play user can chose from over 1.6 million applications for Android \cite{statistaAppStore}.
In 2014 Google's marketplace overtook Apple's Appstore, which had a revenue of over 10 billion in 2013, and became the biggest application store on a mobile platform \cite{wiwoValue}.
\newline
The growth has many advantages.
Some time ago developers only considered iOS as a profitable platform and thus most applications were developed for Apple's \gls{os}.
Now, with Android's overwhelming market share they focus heavily Android \cite{businessProfit}.
But there are downsides as well.
The expanding market for Android, offering many high quality applications, also draws the attention of software pirates.
Crackers do not only bypass license mechanisms and offer applications for free.
Redirecting cash flows or distributing malware using plagiates is an lucrative business model as well.
Android developers are aware of the situation \cite{developersPiracy} and express their need to protect their \gls{ip} on platforms like xda-developers \cite{xdaPiracy} or stackoverflow \cite{stackoverflowPiracy}.
Many of the developers having problems with the license verification mechanism name \textit{Lucky Patcher} as one of their biggest problems \cite{stackoverflowLucky}.
\newline
\newline
The scope of this thesis is to analyse Android cracking applications, like Lucky Patcher,  and to investigate in countermeasures for developers.
