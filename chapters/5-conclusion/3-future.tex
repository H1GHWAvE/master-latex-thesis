\section{Future Work}\label{section:conclusion-future}
There are several developments going on that will change the security landscape in the Android world.
\newline
Every version of Android closes several vulnerabilities but also opens new ones.
Sometimes, the whole architecture is updated, e.g. the introduction of \gls{art}.
With regards to \gls{luckypatcherg}, \gls{art} has the ability to do away with dex bytecode entirely, but in the real world, compatibility needs to be observed and non \gls{art} \gls{apk}s need to be supported for the foreseeable future.
This means \gls{apk}s cannot omit \gls{dex} files and remain vulnerable.
\newline
At the moment, \gls{se} are not standardized and \textit{universal} support not guaranteed by a reasonable number of devices.
\gls{se}s open whole a range of possibilities which can only be tapped, when \gls{se}s reach a critical level of availability.
\newline
As seen in subsection~\ref{subsection:foundation-android-package}, \gls{apk}s are signed and protected on a certain level by self-signed certificates.
Between this level and a walled garden, e.g. total control by Google, there has to be a viable trade off, making use of certification and certification authorities.
