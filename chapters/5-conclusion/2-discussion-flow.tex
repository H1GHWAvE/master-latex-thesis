\section{Discussion}\label{section:conclusion-discussion}
%START TEXT INPUT
This is my real text! Rest might be copied or not be checked!
%START TEXT INPUT

%
clear in beginnign that lvl not sufficiently safe with current technology
unclear degree and fixavle

shortly after start insufficient reilience against reverse engineering, not explusivly to lvl
thus shift from lvl protection to general protection against reverse engineering, decompilation and patching

eternal arms race
no winning solution against all cases, jsut small pieces quantitative improvement
no qualitatively improve resilience
limited to quantitative resilience, matter of time until small steps
generate more work for reengineering, ggf lower motivation for cracker
only matter of time until patching tools catch up, completely new protection schemes need to be devised to counter those
\cite{munteanLicense}
%

%
not a question of if but of when
bytecode tool to generate the licens elibrary on the fly, using random  permutations and injecting it everywhere into the bytecode
with an open platform we have to accept a crack will happen
\cite{digipomLvl}
%


um das ganze zu umgehen content driven, a la spotify, jedoch ist dies nicht mit jeder geschäftsidee machbar

alles hilft gegen lucky patcher auf den ersten blick, jedoch custom patches, welche LuckyPatcher anbietet\cite{munteanLicense}, können es einfach umgehen,
deswegen hilft nur reengineering schwerer zu machen
viele piraten sind nicht mehr motiviert wenn es zu schwer ist\newline
every new layer of obfuscation/modifcation adds another level complexity\newline

solange keine bessere lösung vorhanden unique machen um custom analysis und reengineering zu enforcen und dann
viele kleine teile um die schwierigkeit des reengineeren und angriffs zu erschweren und viel zeit in anspruch zu nehmen um die motivation der angreifer zu verringern und somit die app zu schützen
