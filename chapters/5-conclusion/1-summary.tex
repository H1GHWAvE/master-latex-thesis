\section{Summary}\label{section:conclusion-summary}
The scope of this thesis was to analyse how \gls{luckypatcherg} is carrying out the attack on the license verification libraries and what countermeasures developers can apply to protect their application against it.
\newline
\newline
The first chapter starts with the introduction of software licensing, its goals and the reason it is enforced.
The current situation and problems with licensing on Android is portrayed.
Different approaches to improve and enforce license verification are presented in the related work.
\newline
\newline
The second chapter explains the fundamentals needed to understand why software piracy is a problem.
Android and the steps needed to run an application are explained.
Then the license verification libraries, which are target of \gls{luckypatcherg}, are introduced.
Finally the tools used for the code analysis are portrayed.
\newline
\newline
The third chapter is all about the Android cracking application \gls{luckypatcherg}.
First the functionality is presented.
Then the selection process for the approach is discussed and explained in the black box analysis.
The findings of the black box analysis are presented in the following and abstracted.
\newline
\newline
The fourth chapter suggests two different types of countermeasures.
The first part is about improvements to the current state of the license verification libraries.
The second part introduces a content driven approach to ditch the license verification process.
