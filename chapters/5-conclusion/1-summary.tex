\section{Summary}\label{section:conclusion-summary}
The scope of this thesis was to analyse how \gls{luckypatcherg} is carrying out the attack on the license verification libraries and what countermeasures developers can apply to protect their application against it.
\newline
\newline
The first chapter starts with the introducion of software licensing, its goals and the reason it is enforced.
The current situation and problems with licensing on Android is portrayed.
Different approaches to improve and enforce license verification are presented in the related work.
\newline
The second chapter explains the fundamentals needed to understand why software piracy is a problem.
Android and the steps needed to run an application are explained.
This chapter introduces the license verification libraries which are target of \gls{luckypatcherg} and tools used for the analysis.
\newline
The third patcher is all about the Android cracking tool \gls{luckypatcherg}.
First the functionality is presented.
Then an analysis of the application itself is done followed by a blackbox analysis and the evaluation of the result.
In the end the lessons learned from the analysis are pointed out.
\newline
The fourth chapter suggests three different types of countermeasures.
The first part is about improvements to the current state of the license verification libraries and the addition of integrity checks.
The second part introduces outsourcing of content and encryption as a non predictable implementation of license enforcement.
The third part suggests improvements in the environment to protect against cracking tools.
