\section{Discussion} \label{section:conclusion-discussion}

as long as self signing is allowed, applications can be changes, signed and installed, but google does not want a walled garden as apple on iOS, allowing only applications they approve\cite{codeSigning} \cite{androidSigning} walled garden


clear in beginnign that lvl not sufficiently safe with current technology
unclear degree and fixavle

shortly after start insufficient reilience against reverse engineering, not explusivly to lvl
thus shift from lvl protection to general protection against reverse engineering, decompilation and patching

eternal arms race
no winning solution against all cases, jsut small pieces quantitative improvement
no qualitatively improve resilience
limited to quantitative resilience, matter of time until small steps
generate more work for reengineering, ggf lower motivation for cracker
only matter of time until patching tools catch up, completely new protection schemes need to be devised to counter those
\cite{munteanLicense}
%
research and also a valuable market for companies\newline
Because source code can be easier recovered from an application in comparison
to x86, there is a strong need for code protection and adoption of existing reverse engineering methods. Main parts of Android application functionalities are realized in Dalvik bytecode. So Dalvik bytecode is of main interest for this topic
\cite{schulzLabCourse}
%
not a question of if but of when
bytecode tool to generate the licens elibrary on the fly, using random  permutations and injecting it everywhere into the bytecode
with an open platform we have to accept a crack will happen
\cite{digipomLvl}
%


um das ganze zu umgehen content driven, a la spotify, jedoch ist dies nicht mit jeder geschäftsidee machbar

alles hilft gegen lucky patcher auf den ersten blick, jedoch custom patches, welche Lucky Patcher anbietet\cite{munteanLicense}, können es einfach umgehen,
deswegen hilft nur reengineering schwerer zu machen
viele piraten sind nicht mehr motiviert wenn es zu schwer ist\newline
every new layer of obfuscation/modifcation adds another level complexity\newline

solange keine bessere lösung vorhanden unique machen um custom analysis und reengineering zu enforcen und dann
viele kleine teile um die schwierigkeit des reengineeren und angriffs zu erschweren und viel zeit in anspruch zu nehmen um die motivation der angreifer zu verringern und somit die app zu schützen

close down free installations


Es muss generell immer abgewogen werden zwischen Reichweite und Sicherheit. Von Output den Lucky Patcher gibt, sind die auto patching modes für Google, Amazon und Samsung, die großen Player. Ein Developer muss seine App dort anbieten um Aufmerksamkeit zu bekommen. Deswegen sind diese Stores auch so gut "maintained" von Lucky Patcher.
Im Falle, dass ein Developer "Sicherheit" vorzieht und seine App in einem alternativen Store anbietet, gibt es zwei Scenarien. Entweder entwickelt jemand einen Custom Patch (dex oder native Angriff) wenn ein "allgemeines Interesse" besteht oder die App ist uninteressant und erhält keine Aufmerksamkeit, weder von LP noch Kunden.
Nur weil ein kopeirschutz nicht gekackt ist heißt es noch ncith dass er nciht knackbar ist, sobald genügend interesse besteht wird es jemand versuchen
