\section{Discussion} \label{section:conclusion-discussion}
The chosen black box approach is effective, focusing the view on the one vulnerability attacked - single verification decisions that can be voided to be unary.
\newline
\newline
There are several levels of the race of arms between developers protecting and pirates stealing \gls{ip}.
\newline
On the lowest level, attack vectors of a specific search pattern can be blocked, but \gls{luckypatcherg} will implement an answer and the general problem of the unary vulnerability is not fixed.
\newline
On another level, dex bytecode is always reengineerable, opening a plethora of attack vectors.
This vulnerability can be fixed by replacing bytecode with native code, but native code will be attacked in different ways.
\newline
Even higher concepts of protection will be cracked as part of the race of arms, e.g. encryption, \gls{se}s or \gls{tee}s.
\newline
\newline
Security is always part of a trade off against openness.
For instance, the prevention of tampering \gls{apk}s with self-signed signatures could be improved by introducing a certification authority.
Thinking this further would lead to a \textit{walled garden}, contradicting Android’s philosophy of openness.
