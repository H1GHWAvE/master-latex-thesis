\section{Cracking License Verification} \label{section:luckypatcher-modi}
%START TEXT INPUT
This is my real text! Rest might be copied or not be checked!
%START TEXT INPUT



%
patterns und patching modes grob erklären (modi von luckypatcher die verschiedene operationen (pattern) auf app anwenden) => vorgehensweise zur\newline

see figure~\ref{fig:luckyScreen} middle liuckypatcher offeres different sets of methods to remove the license verification
Auto Mode - "The monomal number of patches. Suitable for most applications with simple protection" - uses patterns

Auto Mode (Inversed) - "There are a few differences from the ”Auto mode”. It may help you, if "Auto mode" was unsuccessful." - uses patterns

Other Patches (Extreme Mode!) - "Additional patches (mnay cause instability). Apply only if the other patterns were unsuccessful. Requires internet. Try to use together with ”Auto mode” or ”Auto mode (Inversed)”." - uses patterns

Auto Mode (Amazon Market) - "Removes License Verification for applciations from Amazon Market" - uses patterns

Auto Mode (SamsungApps) - "Removes License Verification for Apps from SamsungApps" - uses patterns  (is now GalaxyApps)


in order to find out what patterns are doing, different apps had to be analysed after patching
the apps chosen were already owned, in addition an app for each license verification model was created, uploaded and installed from the store so the license verification was working

to verify that license check is enabled, each app was extracted from the device using method described in \ref{subsubsection:tools-apk} and installed on a device with a different google account
then for each app a modified apk see figure~\ref{fig:luckyScreen} left, using one modus is created and copied to a computer for further inspection. so for each app there are 5 modified apks now


as example apps to show results Runtastic Pro\cite{runtasticApp}, Version 6.3, the created LicenseTest and Teamspeak 3\cite{teamspeakApp}, Version 3.0.20.2, are chosen

\begin{table}[h]
\centering
\begin{tabular}{llll}
                                             & \multicolumn{3}{c}{Application}             \\
\multicolumn{1}{c|}{Modus}                   & LicenseTester & Runtastic Pro & Teamspeak 3 \\ \hline
\multicolumn{1}{l|}{Purchased}               & yes           & yes           & yes         \\
\multicolumn{1}{l|}{Pirated}                 & no            & no            & no          \\
\multicolumn{1}{l|}{Auto}                    & yes           & yes           & no          \\
\multicolumn{1}{l|}{Auto (Inversed)}         & no            & yes           & no          \\
\multicolumn{1}{l|}{Extreme}                 & no            & yes           & no          \\
\multicolumn{1}{l|}{Auto+Extreme}            & yes           & yes           & no          \\
\multicolumn{1}{l|}{Auto (Inversed)+Extreme} & no            & yes           & no
\end{tabular}
\caption{Functionality for the test apps before and after patching}
\label{table:functionality}
\end{table}

the result after patching the different apps with each modus returned the patterns used by each modus
