\section{Analysis of Patched Applications} \label{section:luckypatcher-operation}
In addition to analysing the reverse engineered source code, an analysis of patched applications is done.
This is done by first patching an application and creating a modified apk as described in section~\ref{section:luckypatcher-explain} followed by reengineering and analysing them according to the methodology shown in section~\ref{section:reengineering}.
The reason for not patching directly on the device is the circumstance that when the patch is applied on the application, the patched class.dex will be stored as odex.
Since the odex is specific to the code and device, it does not reveal the general approach of patching
Since the code is modified directly, a static analysis is sufficient.
\newline
The goal of reverse engineering the code and comparing it to the original application is to see the changes on different levels.
This includes the \gls{dex} level, on which Lucky Patcher works, the smali level, which makes the \gls{dex} code human readable friendly, as well as Java level, on which the functionality chance can be identified.
On each level the modified and original code will be compared using diff to retrieve the changes in an easy way as well as ignoring the unchanged code.
\newline
Before working on applications from the store, an application, including the \gls{lvl} as Google's tutorial describes it \cite{developersLicensingAdding}, is created.
This is done in order to test whether and how Lucky Patcher works on the most basic version.
For Amazon's Kiwi \gls{drm}, the same application, with deactivated \gls{lvl}, is uploaded to Amazon and injected with their \gls{drm}.
In addition to the basic application, other applications which are purchased from the stores are tested in order to identify additional patterns and to analyse the result.
The applications which were approved to be included into the thesis by the developers are FKUpdater \cite{fkupdater}, Freelatics Bodyweight \cite{freelatics} and Solid Explorer \cite{solidexporer} for the \gls{lvl} and A Better Camera \cite{abettercamera} for the Amazon DRM.
The analysis for Samsung is done by using the example application since the library is implemented the same way into all applications used in.
In contrast to Amazon this can assumed because it is done by the developer and not injected when uploading.
\newline
In addition to the analysis, the modified application is installed on different devices to evaluate the crack.
The difference between the devices is the availability of the corresponding store as well as the presence of root and internet connection.
\newline

patterns und patching modes grob erklären (modi von luckypatcher die verschiedene operationen (pattern) auf app anwenden) => vorgehensweise zur\newline

see figure~\ref{fig:luckyScreen} middle liuckypatcher offeres different sets of methods to remove the license verification
Auto Mode - "The monomal number of patches. Suitable for most applications with simple protection" - uses patterns

Auto Mode (Inversed) - "There are a few differences from the ”Auto mode”. It may help you, if "Auto mode" was unsuccessful." - uses patterns

Other Patches (Extreme Mode!) - "Additional patches (mnay cause instability). Apply only if the other patterns were unsuccessful. Requires internet. Try to use together with ”Auto mode” or ”Auto mode (Inversed)”." - uses patterns

Auto Mode (Amazon Market) - "Removes License Verification for applciations from Amazon Market" - uses patterns

Auto Mode (SamsungApps) - "Removes License Verification for Apps from SamsungApps" - uses patterns  (is now GalaxyApps)


in order to find out what patterns are doing, different apps had to be analysed after patching
the apps chosen were already owned, in addition an app for each license verification model was created, uploaded and installed from the store so the license verification was working

to verify that license check is enabled, each app was extracted from the device using method described in \ref{subsection:tools-apk} and installed on a device with a different google account
then for each app a modified apk see figure~\ref{fig:luckyScreen} left, using one modus is created and copied to a computer for further inspection. so for each app there are 5 modified apks now


as example apps to show results Runtastic Pro\cite{runtasticApp}, Version 6.3, the created LicenseTest and Teamspeak 3\cite{teamspeakApp}, Version 3.0.20.2, are chosen

\begin{table}[h]
\centering
\begin{tabular}{llll}
                                             & \multicolumn{3}{c}{Application}             \\
\multicolumn{1}{c|}{Modus}                   & LicenseTester & Runtastic Pro & Teamspeak 3 \\ \hline
\multicolumn{1}{l|}{Purchased}               & yes           & yes           & yes         \\
\multicolumn{1}{l|}{Pirated}                 & no            & no            & no          \\
\multicolumn{1}{l|}{Auto}                    & yes           & yes           & no          \\
\multicolumn{1}{l|}{Auto (Inversed)}         & no            & yes           & no          \\
\multicolumn{1}{l|}{Extreme}                 & no            & yes           & no          \\
\multicolumn{1}{l|}{Auto+Extreme}            & yes           & yes           & no          \\
\multicolumn{1}{l|}{Auto (Inversed)+Extreme} & no            & yes           & no
\end{tabular}
\caption{Functionality for the test apps before and after patching}
\label{table:functionality}
\end{table}

the result after patching the different apps with each modus returned the patterns used by each modus
