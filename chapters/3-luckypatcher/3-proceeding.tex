\section{Blackbox Analysis} \label{section:luckypatcher-blackbox}
One of the difficulties of black box analysis is the definition of the right test cases.
As a reminder, the scope of the analysis in this thesis are the five \textit{universal} modes and two \textit{universal} patches for removing the license verification and creating an modified \gls{apk}.
The five modes and two patches form the first dimension in the variations of test cases.
Different applications will trigger different reactions in the black box and form a second dimension.
\newline
Initially, fourteen applications are selected for the analysis, but quickly it becomes apparent that only a limited number of reactions is triggered.
For every test case run, \gls{luckypatcherg} announces success and failure of each of nine applied patterns and patches.
\newline
\gls{luckypatcherg}’s different modes and patches are described in figure~\ref{fig:luckyScreen} on the right.
\begin{itemize}
\item The \textit{Auto Mode} - \grqq{The minimal number of patches. Suitable for most applications with simple protection.}\grqq
\item \textit{Auto Mode (Inversed)} - \grqq{There are a few differences from the ”Auto mode”. It may help you, if ”Auto mode” was unsuccessful.}\grqq
\item Other Patches (\textit{Extreme Mode!}) - \grqq{Additional patches (may cause instability). Apply only if the other patterns were unsuccessful. Requires internet. Try to use together with ”Auto mode” or ”Auto mode (Inversed)”.}\grqq
\item \textit{Auto Mode (Amazon Market)} - \grqq{Removes License Verification for applications from Amazon Market}\grqq
\item \textit{Auto Mode (SamsungApps\footnote[1]{SamsungApps is called GalaxyApps, see subsection~\ref{section:license-samsung}})} - \grqq{Removes License Verification for Apps from SamsungApps}\grqq
\end{itemize}
The \textit{Extreme Mode!} can either be applied on its own or combined with the the /textit{Auto Mode} or the \textit{Auto Mode (Inversed)}.
This accounts for the five \textit{universal} modes for the \gls{lvl} while the patches for Amazon and Samsung are in addition.
\newline
A reference application with an implementation according to the tutorial of \gls{lvl} \cite{developersLicensingAdding}, is created for the black box test.
This application is called \textit{LicenseTest}.
It gives full control and knowledge as the source code is available and allows an analyse of \gls{luckypatcherg}’s attacks in the most comprehensive way.
When testing Amazon and Samsung, \textit{LicenseTest} is uploaded including the \gls{lvl}.
Of course,  the \gls{lvl} is disabled because it is not supported on those platforms.
For Samsung, \textit{Zirconia} is implemented according to subsection~\ref{section:license-samsung} while Amazon injects \textit{Kiwi} as discussed in subsection~\ref{section:license-amazon}.
Among the other applications are Runtastic Pro\cite{runtasticApp}, version 6.3, and Teamspeak 3\cite{teamspeakApp}, version 3.0.20.2, for the \gls{lvl} and A Better Camera \cite{abettercamera}, version 3.35, for the Amazon DRM.
These apps were chosen since they were already owned by the author and, upon request, the developers approved them to be mentioned in this thesis.
The rest of the applications is not mentioned here, since the approval was not able to be obtained.
\newline
While initially the test was organized by executing each application in each mode, it becomes immediately apparent that each mode triggers specific patterns and patches and they drive the behaviour.
As patterns can be triggered intentionally in a test case, the whole test is finally organized to analyse one pattern at a time.
This will be described in the next section.
