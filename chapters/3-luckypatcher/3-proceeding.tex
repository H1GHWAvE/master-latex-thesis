\section{Blackbox Analysis} \label{section:luckypatcher-blackbox}
The blackbox approach looks at the parts of the code manipulated by \gls{luckypatcherg}.
Lucky Patcher is used on different applications to get a variety of results and identify how the different implementations are attacked.
This is done with applications implementing the three libraries in the scope of this thesis, \gls{lvl}, \textit{Zirconia} and \textit{Kiwi}.
\newline
The applications are patched using the \textit{Auto Modes}.
The approach for creating a modified \gls{apk} was chosen since \gls{odex} files are device specific and cannot be used for a general conclusion of an attack (see subsection~\ref{subsection:android-dex}).
The outcome is analysed using the methodologies described in  section~\ref{section:reengineering}.
Since only the bytecode is modified, a static analysis is sufficient.
\newline
The goal of reverse engineering the code and comparing it to the original application is to pinpoint changes and analyse them on different abstraction levels.
This includes the \gls{dex} level, on which Lucky Patcher works, the smali level, which makes the \gls{dex} code human readable and the Java level, as the highest level of abstraction and human readability.
On each level, the modified and original code are compared.
The utility diff is used to retrieve the changes in an easy way, ignoring the unchanged code.
\newline
\newline
Besides circumventing the Google \gls{lvl}, \gls{luckypatcherg} supports the the voiding of Amazon's and Samsung's license verification library.
A reference application with known source code and an implementation, according to the tutorial of \gls{lvl} \cite{developersLicensingAdding}, is created in order to have a reference application.
This application is called \textit{LicenseTest}.
It gives full control and knowledge of the code and allows the analyse of \gls{luckypatcherg} attack on the most comprehensive version.
The same application, including the Google \gls{lvl}, is uploaded to the Play Store, Amazon and Samsung. Of course, for Amazon and Samsung, the \gls{lvl} has to be disabled because it is not supported on those platforms.
For Samsung, \textit{Zirconia} is implemented according to subsection~\ref{section:license-samsung} while Amazon injects \textit{Kiwi} as discussed in subsection~\ref{section:license-amazon}.
\newline
To see how \gls{luckypatcherg} handles different implementations, other applications than \gls{luckypatcherg} were analysed as well.
The applications, are Runtastic Pro\cite{runtasticApp}, version 6.3, and Teamspeak 3\cite{teamspeakApp}, version 3.0.20.2, for the \gls{lvl} and A Better Camera \cite{abettercamera}, version 3.35, for the Amazon DRM.
These apps were chosen since they were already owned and approved to be included into the thesis by their developers.
They include Google’s \gls{lvl} and Amazon \gls{drm}.
The analysis for Samsung is done by using only one application.
The \gls{jar} of \textit{Zirconia} cannot be changed by the developer and thus has to be implemented the same way into all applications.
\newline
Run time tests were done on the author's device as well as several others in order to proof LP's ability to target different configurations, e.g. Android version or \textit{root}.
This test is necessary since the modified application can be installed on any device.
It verifieds whether the crack works even though the corresponding store, root or internet connection are not available.
\newline
As described before, \gls{luckypatcherg} offers different modes to patch applications.
Each mode offers a set of patterns to be used of which each change a piece of binary.
These patterns are shown in figure~\ref{fig:luckyScreen} on the right.
A pattern is identified by a name (N1, N2, …) and consists of a set of predefined sequences of bytecode in which certain values are modified.
For the black box test each mode is applied in order to discover patterns applied.
\newline
These are the different modes and what Lucky Patcher describes them as.
\begin{itemize}
\item The Auto Mode - "The minimal number of patches. Suitable for most applications with simple protection".
\item Auto Mode (Inversed) "There are a few differences from the ”Auto mode”. It may help you, if ”Auto mode” was unsuccessful."
\item Other Patches (Extreme Mode!) - "Additional patches (may cause instability). Apply only if the other patterns were unsuccessful. Requires internet. Try to use together with ”Auto mode” or ”Auto mode (Inversed)”."
\item Auto Mode (Amazon Market) - "Removes License Verification for applications from Amazon Market"
\item Auto Mode (SamsungApps) - "Removes License Verification for Apps from SamsungApps" (Note: SamsungApps is called GalaxyApps, see subsection~\ref{section:license-samsung})
\end{itemize}
