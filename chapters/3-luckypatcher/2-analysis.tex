\section{Code Analysis} \label{section:luckypatcher-analysis}
The code analysis was done using Lucky Patcher in version 6.0.4 using the two tools (Androguard and JADX) described in subsection~\ref{subsection:forensics-tools-java}.
The reversed code was inspected using a text editor like Atom \cite{atom}.
\newline
Before analysing the code, a look is taken at the structure.
When loading the folder of the reengineered code into the editor a lot of different code folders can be spotted.
These code folders hold the packages of Lucky Patcher itself. libraries used and resources.
The libraries can be divided into four categories.
\begin{enumerate}
\item Android Support Library v4 with many of it's modules.
It is used for downward compatibility of Android related functions of Lucky Patcher itself.
\item Lucky Patcher code.
They are located in two places.
Utility functions, like "copy file" and "rename", are stored in the package com.chelpus utility functions
The \textit{application code} itself can be found in the package \textit{com.android.vending.billing.InAppBillingService.LACK}.
It contains the activities and functions which are used for cracking applications.
\item The third category are support libraries required by Lucky Patcher to apply it's cracking mechanisms.
This includes libraries, e.g. axml \cite{axml} for serializing the AndroidManifest.xml from Android binary into an ASCII formatted, human readable xml and zip4j \cite{zip4j}, a Java library to handle ZIP files.
The fourth and last category is a modified billing and license library.
It is applied in combination with a proxy to redirect inapp billing and licensing calls.
\item Already cracked \gls{lvl}.
\end{enumerate}
Besides the four code folders, the asset folde stores different predefined custom patches which can be applied applications \cite{munteanLicense}.
\newline
The code itself
