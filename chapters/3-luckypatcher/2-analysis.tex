\section{Code Analysis} \label{section:luckypatcher-analysis}
The code analysis was done using Lucky Patcher in version 6.0.4 using the two tools (Androguard and JADX) described in subsection~\ref{subsection:forensics-tools-java}.
The reversed code was inspected using a text editor like Atom \cite{atom}.
\newline
Before analysing the code, a look is taken at the structure.
When loading the folder of the reengineered code into the editor a lot of different code folders can be spotted.
These code folders hold the packages of Lucky Patcher itself. libraries used and resources.
The libraries can be divided into four categories.
\begin{enumerate}
\item Android Support Library v4 with many of it's modules.
It is used for downward compatibility of Android related functions of Lucky Patcher itself.
\item Lucky Patcher code.
They are located in two places.
Utility functions, like "copy file" and "rename", are stored in the package com.chelpus utility functions
The \textit{application code} itself can be found in the package \textit{com.android.vending.billing.InAppBillingService.LACK}.
It contains the activities and functions which are used for cracking applications.
\item The third category are support libraries required by Lucky Patcher to apply it's cracking mechanisms.
This includes libraries, e.g. axml \cite{axml} for serializing the AndroidManifest.xml from Android binary into an ASCII formatted, human readable xml and zip4j \cite{zip4j}, a Java library to handle ZIP files.
The fourth and last category is a modified billing and license library.
It is applied in combination with a proxy to redirect inapp billing and licensing calls.
\item Already cracked \gls{lvl}.
\end{enumerate}
Besides the four code folders, the asset folde stores different predefined custom patches which can be applied applications \cite{munteanLicense}.
\newline
The analysis is difficult since there is no folder structure sorting the classes in activities or models.
The \textit{patchActivity} in the package \textit{com.android.vending.billing.InAppBillingService.LACK} can be identified as launcher method when looking into the AndroidManifest.xml file.
The folder embracing the \textit{patchActivity} has a flat hierarchy and is not grouped into sub-folders by activities or models.
In the \textit{launcherActivity}, the \textit{listAppsFragment} is created.
It implements most of the app logic in over 17.000 lines of code.
The additional files are models and classes implementing functions, functions like \textit{CorePatch} for writing the output file.
The lack of structure is deliberately chosen to make it hard to get a clear understanding of how exactly Lucky Patcher is patching the attacked application.
Another technique chosen by \gls{luckypatcherg} is obfuscation of strings and methods which makes it impossible to get their purpose just from the name.
At some points it is even impossible for the decompiler tool to disassemble to Java code, forcing it to just return the instructions dump.
\newline
Lucky Patcher has already has already been analysed.
Some of its functions are documented in in Marius-Nicolae Muntean's Master's thesis \textit{Improving License Verification in Android} \cite{munteanLicense}.
