\section{Patch Patterns} \label{section:luckypatcher-patterns}
In order to identify the changes done by each individual \textit{Patch Pattern}, the code of the original and cracked \gls{apk} are compared.
The changes are inspected on dex, smali and Java level with the tools explained in section~\ref{section:reengineering}.
\newline
The names \gls{luckypatcherg} assigns to the \textit{Patch Patterns} and Patches are taken from the patching result output in figure~\ref{fig:luckyScreen} on the right.
While \gls{luckypatcherg} provides seven different patterns for the \gls{lvl}, called \textit{Patch Pattern} N1..N7, it gives to Amazon and Samsung one patch each and calls them Amazon Market Patch and SamsungApps Patch, respectively.
\newline
When patching the \gls{lvl}, \gls{luckypatcherg}'s choice of patterns is dependent on the selected modus.
Table~\ref{table:patterns} gives an overview based on the black box test.
It shows the patching modes and the \textit{Patch Patterns} or \textit{Patches} applied.
\newline
\begin{table}[h]
\centering
\begin{tabular}{l|cccccccccc}
                           & \multicolumn{10}{c}{Patterns}           \\
\multicolumn{1}{c|}{Modus} & N1 & N2 & N3 & N3i & N4 & N5 & N6 & N7 & A & S \\ \hline
Auto                       & X  & X  & X  &     & X  &    &    &  &&  \\
Auto (Inversed)            & X  & X  &    & X   & X  &    &    &  &&  \\
Extreme                    &    &    &    &     &    & X  & X  & X && \\
Auto+Extreme               & X  & X  & X  &     & X  & X  & X  & X && \\
Auto (Inversed)+Extreme    & X  & X  &    & X   & X  & X  & X  &X &&\\
Amazon               &   &  X&   &     &   &   &   &  &X&\\
Samsung               &   &   &   &     &   &   &   &   &&X
\end{tabular}
\caption{Overview of \textit{Patch Patterns}/\textit{Patches} applied by each mode}
\label{table:patterns}
\end{table}
As shown in the following, the changes observed in the black box test are only replacements of bytecode and very limited in scope and number.
They do not add or remove bytecode, i.e. they do not add or remove logic blocks.
Instead \gls{luckypatcherg} enforces a certain control flow, by forcing an evaluation to be \textit{true} or \textit{false} as needed and ignoring actual results of method calls.
This is done by manipulating a single instruction, either opcode or argument.
The surrounding context remains untouched.
\newline
All this is done on the dex bytecode level which opens the question on how the target to be changed is located.
As will be shown, \gls{luckypatcherg} uses bytecode search patterns to locate where the change should be placed and bytecode replace pattern to execute it.
\newline
The search pattern consists of the target instruction to be manipulated and the context, while the replace pattern has the same context with the target instruction changed.
A look at the \gls{luckypatcherg} sources shows that it contains strings, which are formatted like bytecode, include the target instructions and the context in a masked form.
These strings are located in the Java class \textit{com/chelpus/root/utils/odexrunpatch.java} and appear in the context of the \textit{Patch Pattern} they belong to.
Each \textit{Patch Pattern} can have multiple byte code patterns, implementing the same change in different contexts, which are tried when the \textit{Patch Pattern} is used.
\newline
The context in a search pattern is given as a mask of a fixed length with fix points given as explicit byte code tuples and placeholders.
When \gls{luckypatcherg} tries to position a search pattern, it searches for a sequence of bytecodes matching the fix points and the target instruction as given by the mask.
On success, it substitutes that sequence with the replace pattern.
\newline
\newline
\lstinputlisting[
  style=diff,
  breakatwhitespace=false,
  breaklines=true,
  captionpos=b,
  frame=single,
  linerange={65-66},
  caption={Bytecode patterns - Target instructions are colored and placeholders given as \#\#},
  label={codeSnippet:example}
]{data/patterns.diff}
Before explaining the patterns in detail, additional information has to be provided.
In the \gls{dex} file analysis, a simplified presentation of the binary data as as \textit{0a} instead of hexadecimal values like \textit{0x0a} is chosen for improved overview in the diff files.
When converting \gls{dex} files to smali files, the arguments of the opcodes are transferred to variables, e.g. \textit{x} in dex code is v\textit{x} in smali.
\newline
\newline
When the dex code of an application is modified, the checksum has to be recalculated and the file has to be signed again.
These changes can be seen in the diff of the dex files but not explicitly shown in this text.
They are required by Android to make the \gls{dex} file valid but do not change the application logic.

\subsubsection{\textit{Patch Pattern} N1}
Pattern N1 is present in all patching modes, except the solo extreme mode.
It targets the \textit{verify()} method of the \textit{LicenseValidator} class in the \textit{com/google/android/vending/licensing/} folder.
This method is responsible for decrypting and verifying the response from the license server \cite{developersLicensingReference}.
\newline
In can be seen in the dex code in code snippet~\ref{codeSnippet:n1DiffDex} that \textit{1a} and \textit{0f} are swapped in their order.
\newline
\lstinputlisting[
  style=diff,
  breakatwhitespace=false,
  breaklines=true,
  captionpos=b,
  frame=single,
  linerange={3-5},
  caption={Diff on Dex level for N1 pattern},
  label={codeSnippet:n1DiffDex}
]{data/patterns.diff}
When analysing \gls{luckypatcherg}, no bytecode pattern for pattern N1 is identified, neither by name nor by verification.
\newline
When looking at the smali code, the two variables can be identified as blocks of a switch statement.
Due to the internal mapping by the language, variables have different names.
The swap of switch cases \textit{0x1} und \textit{0x2} can be seen in the diff of code snippet~\ref{codeSnippet:n1DiffSmali}.
\newline
\lstinputlisting[
  style=diff,
  breakatwhitespace=false,
  breaklines=true,
  captionpos=b,
  frame=single,
  linerange={5-9},
  caption={Diff on Smali level for N1 pattern},
  label={codeSnippet:n1DiffSmali}
]{data/n1.diff}
In the Java code snippet~\ref{codeSnippet:n1DiffJava}, the changes can be seen in their context.
Before the patch, \textit{LICENSED} and \textit{LICENSED\_OLD\_KEY} both were handled as valid, since \textit{LICENSED} jumps into the next case.
After the patch, \textit{NOT\_LICENSED} starts where \textit{LICENSED\_OLD\_KEY} started before.
Now, \textit{LICENSED} and \textit{NOT\_LICENSED} have the same behavior which means even though the response code is \textit{NOT\_LICENSED}, it is treated as valid.
\newline
\lstinputlisting[
  style=diff,
  breakatwhitespace=false,
  breaklines=true,
  captionpos=b,
  frame=single,
  linerange={11-16},
  caption={Diff on Java level for N1 pattern},
  label={codeSnippet:n1DiffJava}
]{data/n1.diff}
The result is the voiding of the \textit{verify()} switch case.
Despite the input, it always handles it as if the user is verified.

\subsubsection{\textit{Patch Pattern} N2}
Like pattern N1, pattern N2 is applied in all patching modes, except the solo extreme mode.
It is more aggressive, since it does not only attack the \gls{lvl} library, but extends it attacks to other Google Mobile Service (\textit{gms}) libraries, e.g.  \textit{com/google/android/gms/ads/}, as well.
The extended analysis of different applications shows custom libraries are attacked as well.
An example is AnjLab's inapp billing library \cite{inappBilling} of FKUpdater, located at \textit{com/anjlab/android/iab/v3/Security}.
The library contains code for the Google in app billing.
The pattern is applied to other locations to counter a moved \gls{lvl}.
Similar to pattern N1, pattern N2 targets the \textit{LicenseValidator} class's \textit{verify()} method.
\newline
The changes in the \gls{dex} file can be seen in code snippet~\ref{codeSnippet:n2DiffDex}.
The instruction \textit{0a 05} is replaced by \textit{12 15}.
\newline
The search for bytecode patterns used by pattern N2 return one bytecode pattern as seen in code snippet~\ref{codeSnippet:n2patterns}.
\newline
\lstinputlisting[
  style=diff,
  breakatwhitespace=false,
  breaklines=true,
  captionpos=b,
  frame=single,
  linerange={1-7},
  caption={Diff on Dex level for N2 pattern},
  label={codeSnippet:n2DiffDex}
]{data/n2.diff}
The smali in code snippet~\ref{codeSnippet:n2DiffSmali} names the target opcode.
Instead of moving the result of the preceding function to variable \textit{v5}, it is always set to \textit{true}.
\newline
\lstinputlisting[
  style=diff,
  breakatwhitespace=false,
  breaklines=true,
  captionpos=b,
  frame=single,
  linerange={13-15},
  caption={Diff on Smali level for n2 pattern},
  label={codeSnippet:n2DiffSmali}
]{data/n2.diff}
The result in the Java code (see code snippet~\ref{codeSnippet:n2DiffJava}) shows more than just the setting of a variable to \textit{true}.
Instead of proceeding according to the result of the verification of the signature, the result is ignored and the execution is continued inside the condition.
The Java code looks different since the decompiler collapses the \textit{if(true)} statement.
\newline
\lstinputlisting[
  style=diff,
  breakatwhitespace=false,
  breaklines=true,
  captionpos=b,
  frame=single,
  linerange={17-19},
  caption={Diff on Java level for N2 pattern},
  label={codeSnippet:n2DiffJava}
]{data/n2.diff}
The consequence is that the result of the signature validation is ignored.
\textit{verify()} is continued as if the signature was valid.

\subsubsection{\textit{Patch Pattern} N3}
Pattern N3 is different than the other patterns, since there are two versions of it.
The first version, N3, is used in auto mode while pattern N3i is used in the inversed auto mode.
The name N3i is chosen since it is used in the \textit{inversed} mode.
LuckyPatcher groups both patterns as patter N3 since they attack the same mechanism.
They are combined under the same number since both attack the same part of code inside the classes defining the policies.
In case of the basic implementation of the \gls{lvl}, the classes attacked are the \textit{APKExpansionPolicy} and \textit{ServerManagedPolicy} in the  \textit{com/google/android/vending/licensing/} folder.
The two classes are examples of policies offered by Google \cite{developersLicensingReference}.
Pattern N3 attacks their \textit{allowAccess()} method.
\newline
In case of pattern N3, \textit{01} is replaced with \textit{11}, while in case of N3i \textit{11} is replaced by \textit{01} (see code snippet~\ref{codeSnippet:n3DiffDex}).
\newline
\lstinputlisting[
  style=diff,
  breakatwhitespace=false,
  breaklines=true,
  captionpos=b,
  frame=single,
  linerange={1-7},
  caption={Diff on Dex level for N3 pattern},
  label={codeSnippet:n3DiffDex}
]{data/n3.diff}
The bytecode patterns that are used by pattern N3.
They can be divided into three categories.
The first categories are the auto mode bytecode patterns, containing the bytecode \textit{S1} in the replacement pattern.
The second categories are the inverse auto patching mode bytecode patterns, replacing with bytecode \textit{S2}.
The third category the patterns which cannot be assigned to one of these categories.
They contain the variables \textit{R0} and \textit{W0}
\newline
As seen in the dex analysis, the bytecode is not replaced completely, but the source argument is altered.
Category one and two patch the 4 bit value using variable S\textit{x} while the bytecode patterns of the third category are patching the same logic but for different implementations.
The first pattern of category three is used for implementations when the variable is initialized  with a 16 bit source value (\textit{13 \#\# \#\# \#\#}) instead of a 4 bit source value (\textit{12 \#\#}).
The second pattern of category three is applied in implementations where the variable is initialized by getting the the value from an object (\textit{44 \#\# \#\# \#\#}).
\newline
\lstinputlisting[
  style=diff,
  breakatwhitespace=false,
  breaklines=true,
  captionpos=b,
  frame=single,
  linerange={7-38},
  caption={Bytecode patterns for N2},
  label={codeSnippet:n3patterns}
]{data/patterns.diff}
When looking at the smali diff in code snippet~\ref{codeSnippet:n3DiffDex}, the dex code is translated to the initialization of \textit{v1}.
While N3 sets \textit{v1} to \text{1}, N3i sets \textit{v1} to \textit{0}.
\newline
\lstinputlisting[
  style=diff,
  breakatwhitespace=false,
  breaklines=true,
  captionpos=b,
  frame=single,
  linerange={9-15},
  caption={Diff on Smali level for N3 pattern},
  label={codeSnippet:n3DiffSmali}
]{data/n3.diff}
The resulting Java code is shown in code snippet~\ref{codeSnippet:n3DiffJava}.
The pattern enforces the result of the \textit{allowAccess()} method by already initializing the result value with the desired outcome.
While N3 is targeted towards code where the default return value is \textit{false}, pattern N3i is used when the default value is \textit{true}.
\newline
\lstinputlisting[
  style=diff,
  breakatwhitespace=false,
  breaklines=true,
  captionpos=b,
  float=h,
  frame=single,
  linerange={17-25},
  caption={Diff on Java level for N3 pattern},
  label={codeSnippet:n3DiffJava}
]{data/n3.diff}
Both patterns attack the class's \textit{allowAccess()} method which evaluates whether the verification result is according to the policy or not.
The return variable is initiated with the denying result.
It is changed when the evaluation of the input is successfully verified.
Otherwise it stays the same and thus denies the access.
The pattern changes the default value to the allowing result.
This makes the result of the evaluation negligible since the desired result is already set.
There are two versions of the pattern to counter inversed logic which can be inplemented easily.

\subsubsection{\textit{Patch Pattern} N4}
Pattern N4 was only applied once in the test sample and is part of the auto and auto inverse patching modes.
The target of the pattern is the \textit{LicenseChecker} class of the \gls{lvl}.
It is responsible for initializing the license check in its \textit{checkAccess()} method \cite{developersLicensingReference}.
\newline
As seen in code snippet~\ref{codeSnippet:n4DiffDex}, it replaces \textit{38} with \textit{33}.
\newline
\lstinputlisting[
  style=diff,
  breakatwhitespace=false,
  breaklines=true,
  captionpos=b,
  frame=single,
  linerange={1-3},
  caption={Diff on Dex level for N4 patch},
  label={codeSnippet:n4DiffDex}
]{data/n4.diff}
asdasda
\newline
\lstinputlisting[
  style=diff,
  breakatwhitespace=false,
  breaklines=true,
  captionpos=b,
  frame=single,
  linerange={40-42},
  caption={Bytecode patterns for N4},
  label={codeSnippet:n4patterns}
]{data/patterns.diff}
In the smali code snippet~\ref{codeSnippet:n4DiffSmali} this change can be identified as replacing \textit{if-eqz} with \textit{if-ne}.
The opcode \textit{if-eqz} takes one argument \textit{v0} while \textit{if-ne} takes two arguments.
Since the value is \textit{0} in the \gls{dex} file, it is interpreted as \textit{v0} for the second argument.
\newline
\lstinputlisting[
  style=diff,
  breakatwhitespace=false,
  breaklines=true,
  captionpos=b,
  frame=single,
  linerange={5-7},
  caption={Diff on Smali level for N4 patch},
  label={codeSnippet:n4DiffSmali}
]{data/n4.diff}
The Java code in code snippet~\ref{codeSnippet:n4DiffSmali} shows the result of the change.
In the original code, the result of \textit{mPolicy.allow()} was checked.
In case the policy did not allow to continue, i.e. the result was \textit{false}, the condition block was executed.
The changes of the pattern result in the check for inequality of the result of \textit{mPolicy.allow()}.
Since the result of the method is the same and thus the condition is never fulfilled, the condition block is never called.
\newline
\lstinputlisting[
  style=diff,
  breakatwhitespace=false,
  breaklines=true,
  captionpos=b,
  frame=single,
  linerange={9-11},
  caption={Diff on Java level for N4 patch},
  label={codeSnippet:n4DiffJava}
]{data/n4.diff}
The result of patching with pattern N4 is that in the \textit{checkAccess()} method the result, whether the policy allows to continue, is never considered.

\subsubsection{\textit{Patch Pattern} N5}
As part of the extreme mode, pattern N5 changes the \gls{dex} file similar to pattern N2 and targets the \textit{LicenseValidator}'s \textit{verify()} method.
It cannot be applied on the standard implementation of the \gls{lvl}.
\newline
In the Java diff in code snippet~{codeSnippet:n5DiffJava} the result of patching can be seen.
The original code stores the parsing result of the first entry of the array \textit{v2\_1} in an attribute of an object (\textit{v0\_0.a}).
After applying the patch, the entry is still parsed but ignored and \textit{v0\_0.a} is set to \textit{0}.
\newline
\lstinputlisting[
  style=diff,
  breakatwhitespace=false,
  breaklines=true,
  captionpos=b,
  frame=single,
  linerange={1-4},
  caption={Diff on Java level for N5 patch},
  label={codeSnippet:n5DiffJava}
]{data/n5.diff}
The pattern sets the response code of the object to \textit{LICENSED}.
The real reponse code is ignored and the code continues even though the server did not verify the license.

\subsubsection{\textit{Patch Pattern} N6}
Pattern N6 is part of the extreme mode and, similar to the pattern N1, N2 and N5, it attacks the \textit{verify()} method in the \gls{lvl}'s \textit{LicenseValidator} class.
\newline
This pattern changes three values of the \gls{dex} file which can be seen in code snippet~\ref{codeSnippet:n6DiffDex}.
The first changed value is \textit{38} which is replaced by \textit{12}.
The second value is \textit{06} which is replaced by \textit{00}.
The third change is the replacing of \textit{4a} by \textit{00}.
\newline
\lstinputlisting[
  style=diff,
  breakatwhitespace=false,
  breaklines=true,
  captionpos=b,
  frame=single,
  linerange={1-3},
  caption={Diff on Dex level for N6 patch},
  label={codeSnippet:n6DiffDex}
]{data/n6.diff}
The change of the code structure is more visible in the code snippet~\ref{codeSnippet:n6DiffSmali}.
The first results in the initialization of \textit{p2} with \textit{0}.
The second change is required since the opcode \textit{const/4} consists of the  opcode tuple and another tuple of arguments while the original \textit{if-eqz} has the opcode tuple, the argument tuple and a target tuple.
In order to get a valid syntax, the code has to be changed accordingly.
The result are two \textit{nop} oeprations \textit{00 00}, which are interpreted as \textit{nop} and an empty line.
The third change is the replacing arguments \textit{p2} and \textit{v4} of the \textit{if-eq} evaluation with \textit{v0} for each.
\newline
\lstinputlisting[
  style=diff,
  breakatwhitespace=false,
  breaklines=true,
  captionpos=b,
  frame=single,
  linerange={5-12},
  caption={Diff on Smali level for N6 patch},
  label={codeSnippet:n6DiffSmali}
]{data/n6.diff}
The changes of the pattern are presented in code snippet~\ref{codeSnippet:n6DiffJava}.
Instead for checking the reponse code whether it is \textit{LICENSED} it is set to that value.
This results in the constant values inside the if and switch statements.
In addition, the check whether the response code is not \textit{NOT\_LICENSED} is replaces with the check whether it is not \textit{LICENSED}.
\newline
\lstinputlisting[
  style=diff,
  breakatwhitespace=false,
  breaklines=true,
  captionpos=b,
  frame=single,
  linerange={14-19},
  caption={Diff on Java level for N6 patch},
  label={codeSnippet:n6DiffJava}
]{data/n6.diff}
This pattern prevents the execution for cases where the \textit{verify()} has to handle reponse codes that are neither \textit{LICENSED}, \textit{NOT\_LICENSED} or \textit{LICENSED\_OLD\_KEY}.
It voids the impact of those reponse codes.
Instead it proceeds successfully since the reponse code is set to \textit{LICENSED}.

\subsubsection{\textit{Patch Pattern} N7}
The final pattern for the \gls{lvl} is pattern N7, a brute force version of pattern N2.
Inside the lvl, it patches the \textit{ILicenseResultListener} class's \textit{onTransact()} method, which is the implementation for the callback for interprocess communication and receives the async response from the license server \cite{developersLicensingReference}.
In addition to the lvl classes, the pattern is applied to all classes residing in the \url{com/android/} package.

The outcome might not be stable anymore and thus should only applied when the other modes are not successfully .
\newline
Similar to pattern N2, pattern N7 attacks by initializing the variable with \textit{false} instead of moving a result of a method into it.
\newline
\lstinputlisting[
  style=diff,
  breakatwhitespace=false,
  breaklines=true,
  captionpos=b,
  frame=single,
  linerange={1-4},
  caption={Diff on Java level for N7 patch},
  label={codeSnippet:n7DiffJava}
]{data/n7.diff}
The goal of this attack is to patch the \textit{onTransact}, wherever it may be located.
This methods is responsible for calling the \textit{verifyLicense()} implementation of the \textit{ILicenseResultListener}.
Instead of passing the response code from the server, the method is called using the initialized \textit{0}.
This way the server's reponse code is ignored since it is replaced by the code for \textit{LICENSED}.

\subsubsection{Amazon Market Patch}
Amazon does not have different patches or pattern.
Since the \textit{Kiwi} library is injected by Amazon and cannot be customized by the developer, only one patch with pattern is necessary.
The pattern is applied twice while patching.
The first class tageted is \textit{com/amazon/android/licensing/b.java} while the second class is \textit{com/amazon/android/o/d.java}.
While \textit{b.java} is responsible for the verification of the license, \textit{d.java} is reponsible for handling the expiration of the license.
\newline
The Amazon pattern works similar to the \gls{lvl}'s pattern N4 and replaces (\textit{38}) with (\textit{33}).
\newline
\lstinputlisting[
  style=diff,
  breakatwhitespace=false,
  breaklines=true,
  captionpos=b,
  frame=single,
  linerange={1-3},
  caption={Diff on Dex level for Amazon patch},
  label={codeSnippet:amazonDiffDex}
]{data/amazon.diff}
The pattern replaces \textit{if-eqz} with \textit{if-ne} as seen in code snippet~\ref{codeSnippet:n4DiffSmali}.
The opcode \textit{if-eqz} evaluates only the one argument while \textit{if-ne} takes two arguments.
Since the value in the \gls{dex} file, where the second argument of \textit{if-ne} is, is \textit{0}, \textit{v0} and \textit{v0} are compared.
\lstinputlisting[
  style=diff,
  breakatwhitespace=false,
  breaklines=true,
  captionpos=b,
  frame=single,
  linerange={5-7},
  caption={Diff on Smali level for Amazon patch},
  label={codeSnippet:amazonDiffSmali}
]{data/amazon.diff}
The Java code presentation helps to interpret the changes of the attack.
Since the if statement is now always wrong inside the \textit{b.java} class, the code block for repsonse codes not \textit{APPLICATION\_LICENSE} is never called.
The same canges are applied to the \text{d.java} class.
After analysing the dependencies, it can be said that the function checks whether the given string is not null and then returns \textit{true}.
After patching, the result is always \textit{true} since the new check is always false as in \textit{b.java}.
\newline
\lstinputlisting[
  style=diff,
  breakatwhitespace=false,
  breaklines=true,
  captionpos=b,
  frame=single,
  linerange={9-11},
  caption={Diff on Java level for Amazon patch},
  label={codeSnippet:amazonDiffJava}
]{data/amazon.diff}

The analysis of the Amazon patch indicates that there are less patterns needed since code modifications have to be expected.
Patches are applied to manipulate the checks in case the response code is different than \textit{APPLICATION\_LICENSE}.
The result is forced to be always \textit{true} and thus the license verification always passes.


\subsubsection{Samsung Market Patch}
Similiar to the Amazon's \textit{Kiwi} library, Samsung's \textit{Zirconia} library cannot be modified.
For this reason cracking the library requires only one patch.
The patch is applied on the \textit{LicenseRetriever} and \textit{Zirconia} class in the \textit{com/samsung/zirconia} package.
The Samsung patch uses two patterns, called S1 and S2 in order to distinguish between them.
While S1 is applied on both classes once, S2 is applied twice but only on the \textit{Zirconia} class.
\newline
While Pattern S1 replaces \textit{d6} with \textit{00}, pattern S2 uses \textit{12} instead of \textit{0a}.
\newline
\lstinputlisting[
  style=diff,
  breakatwhitespace=false,
  breaklines=true,
  captionpos=b,
  frame=single,
  linerange={1-7},
  caption={Diff on Dex level for Samsung patch},
  label={codeSnippet:samsungDiffDex}
]{data/samsung.diff}
The result of pattern S1 is that the \textit{if-eq} statement now uses \textit{v0} and \textit{v0}.
The result of this comparison is always \textit{true}.
Pattern S2 has the effect that \textit{v0} does not return the result of the preceding method but always \textit{true}.
\newline
\lstinputlisting[
  style=diff,
  breakatwhitespace=false,
  breaklines=true,
  captionpos=b,
  frame=single,
  linerange={9-15},
  caption={Diff on Smali level for Samsung patch},
  label={codeSnippet:samsungDiffSmali}
]{data/samsung.diff}
The presentation in code snippet~\ref{codeSnippet:samsungDiffJava} contains the changes in Java code.
Instead of checking the reponse code for validity, \textit{LicenseRetriever}'s \textit{receiveResponse()} method always skips the check, when pattern S1 is applied, and executes as if it was valid.
In the method \textit{checkerThreadWorker()} of the \textit{Zirconia} class, pattern S1 voids the check of the response code and always continues as if the response code was valid.
\newline
Pattern S2 works on the methods \textit{checkLicenseFile()} and \textit{checkLicenseFilePhase2()} of the \textit{Zirconia} class.
Instead of returning the result of the license check, the methods return always \textit{true}.
\newline
\lstinputlisting[
  style=diff,
  breakatwhitespace=false,
  breaklines=true,
  captionpos=b,
  frame=single,
  linerange={17-24},
  caption={Diff on Java level for Samsung patch},
  label={codeSnippet:samsungDiffJava}
]{data/samsung.diff}
The result of applying the patch is that not only the license file checks are voided and return the verification as \textit{true} on default.
In addition, reponse codes other than \textit{LICENSED} are accepted since they are neither checked for validity nor to the stored one, which should be valid since it was stored.
