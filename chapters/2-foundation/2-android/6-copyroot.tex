\subsection{Copy Protection and Root} \label{subsection:android-copyroot}
%START TEXT INPUT
Now that the application is installed and ready to run.
In order to prevent unauthorized usage of the app copy protection is applied.
After downloading \gls{apk} from an application store it is moved to /data/app on the phone upon installation.
The user has no rights to access this folder and thus not copy the application.
This mechanism has a major flow since when a single user can get hold of the \gls{apk} and redistribute it, copy protection for the application is circumvented.
This was effective in the early days of Android when rooting was not easily facilitated.
\newline
"Rooting" or "getting root" is the process of modifying the operation system's software that shipped with a device in order to get complete control over it.
The name "root" comes from the Linux \gls{os} world and is the user with the most privileges.
This allows the user to overcome limitations by carriers and manufacturers, like removing pre-installed applications, extend system functionality or to upgrade to custom versions of Android.
Manufacturers and carriers do not aprove rooting but since the access is usually gained by exploiting vulnerabilities in the system's code or device drivers, they cannot prevent it.
Vulnerabilites are quite common and documented \cite{androidVulnerabilities}.
Today it is easy, even for non-techies, to gain root.
There are videos and tutorials on the internet, even tools to automate the process, like Wugfresh's Rootkits \cite{wugfresh}, are available.
Rooting is usually bundles with installing a progremm called "su" which manages the root access for applications which request it.
Rooting a phone is not without risk since installing bad files can result in the so called "bricking" meaning the phone is nonfunctional since the software cannot be executed anymore.\cite{androidpoliceRoot}
\newline
Even though rooting brings a lot of improvements to the table it is bad for the copy protection making it non-existent. For this reason new ways of protecting applications have to be invented.
