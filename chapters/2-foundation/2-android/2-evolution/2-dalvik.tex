\subsubsection{Dalvik Virtual Machine} \label{subsubsection:android-evolution-dvm}
%START TEXT INPUT
This is my real text! Rest might be copied or not be checked!
%START TEXT INPUT








AUFBAU/HERKUNFT DALVIK MACHIEN\newline
difference to JVM -> JVM is a stack based machine whereas DVM is register based -> warum, vorteile/nachteile, hisotrie\newline
developed for Android to be more efficient and smaller due to the limited resources on mobile devices -> quelle\newline
AUFBAU DALVIK BYTECODE\newline
The program code of an Android application is delivered in form of Dalvik bytecode\newline
It will be executed by the Dalvik Virtual Machine and is comparable to Java bytecode. So there is a separate
optimizing step while installing an application, where the bytecode gets optimized for the underlaying architecture. The optimized form is also called ”odex”. The optimization is done by a program called ”dexopt” which is part of the Android platform. The DVM can execute optimized as well as not optimized Dalvik bytecode\newline

.dex file
The Dalvik bytecode consists of opcodes and is thus hard to read for humans. The Cracking Tool has to modify the opcodes in order to alter the behaviour of the application. Since it is directly read by the Dalvik virtual machine, it is the single point of truth.\newline

.odex file
Dalvik bytecode of an application is normally not optimized, because it is executed by a DVM which can run under different architectures\newline

\begin{table}[]
\centering
\caption{Dex File Format}
\label{table:foundation-forensic-dex}
\begin{tabular}{|l|l|l|l}
\hline
\multicolumn{2}{|c|}{Magic}             & checksum        & \multicolumn{1}{c|}{}                   \\ \cline{1-3}
\multicolumn{4}{|c|}{signature}                                                                     \\ \hline
File size         & Header size         & Endian Tag      & \multicolumn{1}{l|}{Link size}          \\ \hline
Link offset       & Map offset          & String IDs Size & \multicolumn{1}{l|}{String IDs offeset} \\ \hline
Type IDs Size     & Type IDs offset     & Proto IDs Size  & \multicolumn{1}{l|}{Proto IDs offset}   \\ \hline
Field IDs Size    & Field IDs offset    & Method IDs Size & \multicolumn{1}{l|}{MethodIDs offset}   \\ \hline
Classdef IDs Size & Classdef IDs offset & Data Size       & \multicolumn{1}{l|}{Data offset}        \\ \hline
\end{tabular}
\end{table}

dex format \url{http://source.android.com/devices/tech/dalvik/dalvik-bytecode.html}\newline

ablauf starten von app\newline
When an Android application is executed, the process consists of the following four parts:
• Dalvik bytecode, which is located in the dex file
• Dalvik Virtual Machine [13], which executes the Dalvik bytecode
• Native Code, like shared objects, which is executed by the processor
• Android Application Framework, which provides services for the application\newline
