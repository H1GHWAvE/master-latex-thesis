\subsubsection{Dalvik Virtual Machine} \label{subsubsection:android-evolution-dvm}
%START TEXT INPUT
This is my real text! Rest might be copied or not be checked!
%START TEXT INPUT


%
The applications for Android are written using the Java programming language.

stack abstraction is the Dalvik Virtual Machine (DVM)\newline
DVM is highly tailored to work according to the specifications of the Android platform\newline
optimized for a slower CPU in comparison with a stationary machine andworks with relatively little RAM memory (• limited processor speed
• limited RAM
• no swap space
• battery powered
• diverse set of devices
• sandboxed application runtime)\cite{ehringerDalvik}\newline
DVM is register-based, differing from the standard Java Virtual Machine (JVM) which is stack-based, register-based architectures require fewer executed instructions than stack-based architectures, register-based code is approximately 25 percent larger than the stack-based, the increase in the instructions fetching time is negligible: 1.07 percent extra real machine loads\cite{ehringerDalvik}\newline
the Android OS has no swap space imposing that the virtual machine works without swap. Finally, mobile devices are powered by a battery thus the DVM is optimized to be as energy preserving as possible, Except being highly efficient, the DVM is also designed to be replicated quickly because each application runs within a “sandbox”: a context containing its own instance of the virtual machine assigned a unique Unix user ID\newline

wie der build process funktioniert wird später gesondert beschrieben, hier sagen wir einfach das ergebnis ist die dex datei\newline

AUFBAU DEX
DVM is register based. Registers are considered 32 bits wide to store values such as integers or floating point numbers. Adjacent register pairs are used to store 64-bit values\newline
dest-then-source ordering for its arguments\newline
there are 218 used valid opcodes in Dalvik bytecode -> QUELLE \newline

Due to its simplicity over bytecode for other architectures as well as the little protection applied in practice, Dalvik bytecode is currently an easy target for the reverse engineer.
\cite{kovachevaMaster} \cite{ehringerDalvik}
%




AUFBAU/HERKUNFT DALVIK MACHIEN\newline
difference to JVM -> JVM is a stack based machine whereas DVM is register based -> warum, vorteile/nachteile, hisotrie\newline
developed for Android to be more efficient and smaller due to the limited resources on mobile devices -> quelle\newline
AUFBAU DALVIK BYTECODE\newline
The program code of an Android application is delivered in form of Dalvik bytecode\newline
It will be executed by the Dalvik Virtual Machine and is comparable to Java bytecode. So there is a separate
optimizing step while installing an application, where the bytecode gets optimized for the underlaying architecture. The optimized form is also called ”odex”. The optimization is done by a program called ”dexopt” which is part of the Android platform. The DVM can execute optimized as well as not optimized Dalvik bytecode\newline

unterschied \url{http://newandroidbook.com/files/Andevcon-DEX.pdf}\newline

.dex file
The Dalvik bytecode consists of opcodes and is thus hard to read for humans. The Cracking Tool has to modify the opcodes in order to alter the behaviour of the application. Since it is directly read by the Dalvik virtual machine, it is the single point of truth.\newline

.odex file
Dalvik bytecode of an application is normally not optimized, because it is executed by a DVM which can run under different architectures\newline
