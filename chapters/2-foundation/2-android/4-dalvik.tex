\subsection{Dalvik Virtual Machine} \label{subsection:android-dalvik}
The original \gls{vm} powering Android is the \gls{dvm}.
It was designed by Dan Bornstein and named after an Iclandic town and introduced along with Android in 2008 \cite{developersRelease}.
\newline
In contrast to a stationary computer, a mobile device has a lot of constraints.
Since it is powered by a battery, the processing power and RAM are limited to fit power consumption restraints.
In addition to these hardware limitations, Android has some additional requirements, like no swap for the RAM, the need to run on a diverse set of devices and in a sandboxed application runtime.
In order to deliver best performance and run efficiently, it has to be designed according to these requirements.
\newline
The \gls{dvm} is a customized and optimized version of the \gls{jvm} and based on Apache Harmony.
Even though it is based on Java, it is not fully J2SE or J2ME compatible since it uses 16 bit opcodes and register-based architecture in contrast to the stack-based standard \gls{jvm} with 8 bit opcodes.
The advantage of register-based architecture is that it need less instructions for execution than stack-based architecture which results in less CPU cycles and thus less power consumption which is important for battery driven devices.
The downside of this architecture is the fact that it has an approximatly 25\% larger codebase for the same application and negligible larger fetching times.
In addition to the lower level changes, the \gls{dvm} is optimized for memory sharing.
It stores references bitmaps seperated from objects and optimizes application startup by using zygotes. \cite{ehringerDalvik} \cite{andevconDalvikART}
\newline
The last big change made to the \gls{dvm} was the introduction of \gls{jit}, which will be part of the discussion in subsection~\ref{subsection:android-art}, in Android version 2.2 "Froyo".
