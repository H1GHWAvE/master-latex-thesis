Since the original copy protection of subsection~\ref{subsection:android-copyroot} can be circumvented, a new and secure protection for Android applications is needed.
Now a license verification is done by a central authority, i.e. an application store.
Google, as the main of contributor to Android and provider of its biggest store, started this by providing the \gls{lvl} as described in subsection~\ref{section:license-google}.
\newline
Android’s philosophy is to allow the installation of apps from any source and not only the Google Play, other stores were created to get a piece of Google's Android business.
Some of the most widespread stores are from Amazon and Samsung, both offer a license verification solution as well.
\newline
Amazon does not only have the Amazon Store, but is also trying to create their own ecosystem by selling the \textit{Fire tablets}.
They use a flavor of Android tailored to fit Amazon's needs and come at a low price.
\newline
Samsung pursues a different approach.
In addition to a store, they are also offering different services to bind to their ecosystem.
There are different Chinese and niche stores as well.
Niche stores with few users and offerings are less subject to attack.
Some do not even have any license verification.
\newline
The stores in focus for this thesis have to fight piracy in order make their store attractive for developers.
\newline
\newline
The scope of this thesis are Google, Amazon and Samsung since they have a critical mass in users and a license verification mechanism.
