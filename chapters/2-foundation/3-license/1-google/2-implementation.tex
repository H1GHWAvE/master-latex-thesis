\subsubsection{Implementation of the License Verification Library} \label{section:license-google-implementation}
The provided library of Google can be implemented in only a few steps.
First of all, a Google Publisher Account for Google Play is needed.
It enables the developer to publish applications on Google Play and take advantage of the \gls{lvl}.
As soon as an application is created in the Google Developer Console the public/private key is created.
Each key pair is app specific and can be found under "Services \& API".
It is later on added into the application. \cite{developersLicensingSetup}
\newline
Now that the Google Play prerequisites are set, the \gls{lvl} has to be extracted from the Android \gls{sdk} folder and added to the application.
As soon as the \gls{lvl} is part of the application project the licensing permission can be added to the application's manifest (see code snippet~\ref{codeSnippet:lvlPermission}), else the \gls{lvl} will throw an exception.
\lstinputlisting[
  float=h,
  basicstyle=\footnotesize,
  breakatwhitespace=false,
  breaklines=true,
  captionpos=b,
  frame=single,
  numbers=left,
  language=Java,
  linerange={7-9},
  firstnumber=7,
  caption={Include permission to check the license in AndroidManifest.xml \cite{developersLicensingAdding}},
  label={codeSnippet:lvlPermission}
]{data/permission.xml}
The last step is to create the license call which can be seen in code snippet~\ref{codeSnippet:lvlSetup}.
\lstinputlisting[
  float=h,
  basicstyle=\footnotesize,
  breakatwhitespace=false,
  breaklines=true,
  captionpos=b,
  frame=single,
  numbers=left,
  language=Java,
  linerange={57-64},
  firstnumber=57,
  caption={Setting up the LVL license check call},
  label={codeSnippet:lvlSetup}
]{data/lvl.java}
As described in subsection~\ref{section:license-google-functional}, different information has to be aquired and added to the license request.
In order to identify the user the its unique id is aquired.
The id is created randomly when the user sets up the device for the first time and remains the same for the lifetime of the user's device \cite{androidSecure}.
The obfuscator is later applied on the cached license response data to prevent manipulation or reuse by root user.
In order to decide what should happen to the reponse data and whether it should be stored, a policy is specified.
This can be either one of the provided policies are a custom one.
Before making the license check request, a callback has to be definden, which can be seen in code snipped~\ref{codeSnippet:lvlCallback}.
\lstinputlisting[
  float=h,
  basicstyle=\footnotesize,
  breakatwhitespace=false,
  breaklines=true,
  captionpos=b,
  frame=single,
  numbers=left,
  language=Java,
  linerange={134-150},
  firstnumber=57,
  caption={LVL license check callback},
  label={codeSnippet:lvlCallback}
]{data/lvl.java}
The callback contains actions for the different outcomes of the license check.
The scenarios are either applicationError, e.g. when no connection can be established, dontAllow in case license is not valid and allow when the users status is verified.
Now the license check can be initiated with the values. \cite{developersLicensingSetup}\cite{developersLicensingAdding}
\newline
The license check can be implemented in any activity and executed as the developer likes.
