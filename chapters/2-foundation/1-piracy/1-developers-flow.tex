\subsection{Developers} \label{subsection:foundation-piracy-developers}
Especially for software developers piracy is a problem.
The most apparent issue is clear at first glance. Stealing a developer's \gls{ip} and redistributing without his involvement results in a loss of revenue.
People are then downloading an application either for free or pay the pirate and thus do not generate revenue for the originator.
\newline
At second glance, the problems are even more complex.
Income for the developer is not only lost when the user is not paying for the software, the pirate can also influence the follow up revenue by modifying the application itself.
There are two main types of revenue not generated by the purchase of the application.
The first type is the inapp purchase.
They are a popular source of income for so called freemium applications or lite versions of apps.
In case of the the freemium app, the download is for free and and includes all features. Inapp purchases, e.g. inapp currency, enable the user to proceed faster or to buy cosmetic modifications.
The lite version application is a little bit different. The download is free as well but the application comes with a restricted feature set or limited time of use.
In order to take advantage of the unlocked feature set the user can buy the full license via inapp purchase.
Apps can include a mix and various degrees of theses variants.
Pirates can disable the payments for the features, enabling users to recieve the inapp purchase for free and thus no earnings are generated for the developer.
\newline
The second type of follow up revenue is generated by showing inappAds.
When this feature is included, advertisments are shown inside the application and the developer is payed by the amount of ads seen and clicked by the user.
The Ad Unit ID \cite{googleAdmob} is responsible to assign earnigns generated by an mobile advertisment to the developer.
When an application is pirated, this code can be replaced by the pirate's one. Future revenues generated by advertisments in the application will not be assigned to the developer but redirected to the pirate.
\newline
\newline
Beside monetary issues, additional problems arise when the app is moved to a blackmarket store or website  and distributed without the environment of an official app store.
This results in the loss of control over the application for the developer.
This means the developer can no longer provide the users support or updates for the application to fix crashes caused by malfunctions or security issues.
Users which do not know that they are using a pirated version will connect the unsatisfying behaviour to the developer.
This can result in the loss of future revenues which are not even connected to this application.
In addition to the loss over the application this can cause unpredictable scenarios.
Since the developer cannot monitor the growth of the application over tools which are included in the market place of choice, he can face unpredictable high traffic.
This can stress the server because they were not scaled according to the growth as well as additional costs caused by the traffic which is not covered by additional purchase income \cite{lierschDeveloperThreats}.
\newline
\newline
Developers have to live of their applications.
When they do not earn money, either because the revenue stream is redirected or because their \gls{ip} is stolen and commercialized by someone else, or even lose it due to maintaining costs, they cannot continue with developing and their skills are lost.



%START TEXT INPUT

%on the first view lost revenues when the application is not bought anymore but downloaded from a blackmarket or website
%on the second view the application can not only be made available without purchasing it, but it can also be modified and misused further
%- the attacker can replace the developer's google ad id with his own and this way the ad revenues go to his pockets and are missing in the developers one
%- disable the verification for in app sales, this way the user can buy items from the e.g. inapp store without being charged and the developer not earnign anything, or even more direct: simply activate the pro feature for application in the code

%overall when the app is taken from the environment of the app store it is sold  and moved to blackmarket/website to download, the developer has no more control over it, he cannot support it with fixes and updates
%when this happens or maliscious behaviour was included by the pirate, the bad reputation is on the developer
%also the generated traffic cannot be predicted anymore, this traffic does not make money but costs us, double fail

%and if the code is stolen in general, it can be reverse engineered,  the awesome algorithms included are lost and can be used in any way the pirate likes
%\cite{lierschDeveloperThreats}
