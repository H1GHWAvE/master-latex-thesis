\subsection{Piracy on Android} \label{subsection:foundation-piracy-android}
Piracy is widespread on the Android platform.
Especially in countries like China, piracy is as high as 90\% due to restricted access to Google Play \cite{piracyRate}.
Sources for pirated applications can be easily found on the internet.
A simple search, containing \textit{free apk} and the applications name, returns plenty of results on Google Search.
The links direct the user to black market applications, like Blackmart \cite{blackmartStore}, and websites offering cracked \gls{apk}, such as crackApk \cite{crackApk}.
They claim to be user friendly because they offer older versions of applications.
Their catalog even includes premium apps, which are not free in the Play Store and include license verification mechanisms \cite{apksfree}.
Offering these applications is only possible when the license mechanism is cheated.
Software pirates practice professional theft and expose users to risks (see section~\ref{subsection:foundation-piracy-users}).
\newline
\textit{Today Calendar Pro} is an example for the dimensions piracy can reach for a single application.
The developer stated in a Google+ post that the piracy rate of the application is as high as 85\% on a given day. \cite{xdaPiracy} \cite{developersPiracy}
Since it looks like that license mechanisms are no obstacle for pirates and sometimes cracked within days, some developers do not implement any copy protection \cite{recodeMonument}.
\newline
Android applications are at an especially high risk for piracy because of their use of bytecode, which is an easy target of reverse engineering as shown in the further proceeding.
