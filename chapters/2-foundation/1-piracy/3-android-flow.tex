\subsection{Piracy on Android} \label{subsection:foundation-piracy-android}
\begin{itemize}
    \item test
\end{itemize}
Piracy is widespread on the Android platform. Especially in countries like China, piracy is as high as 90\% due to restricted access to Google Play \cite{piracyRate}.
Sources for pirated applications can be easily found on the internet.
Simple searches containing "free apk" and the applications name return plenty of results on Google Search.
The links direct to black market applications, as Blackmart \cite{blackmartStore}, and websites for cracked \gls{apk}, such as crackApk \cite{crackApk}.
Black market providers claim to be user friendly because they offer older versions of applications.
Their catalog includes premium apps, which are not free in the Play Store and include license verification mechanisms, as well.
This is only possible when the license mechanism is cracked \cite{apksfree}.
They practice professional theft and puts users in danger (see section~\ref{subsection:foundation-piracy-users}).
\newline
\textit{Today Calendar Pro} is an example for the dimensions piracy can reach for a single application.
The developer stated in a Google+ post that the piracy rate of the application is as high as 85\% at the given day \cite{xdaPiracy} \cite{developersPiracy}.
\newline
Since license mechanisms are no obstacle for pirates, some developers do not implement any copy protection at all since it is cracked within days \cite{recodeMonument}.
Android applications are at an especially high risk for piracy because bytecode in general is an easy target to reverse engineer as shown in the further proceeding.
