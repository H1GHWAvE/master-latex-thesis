%START TEXT INPUT
The Cracking Tool has to alter an application's behaviour by applying patches only to the \gls{apk} file, since it is the only source of code on the phone. This is the reason for the investigations to start with analysing the \gls{apk}. This is done using static analysis tools. The aim is to get an accurate overview of how the circumventing of the license verification mechanism is achieved. This knowledge is later used to find counter measurements to prevent the specific Cracking Tool from succeeding.\newline
The reengineering has to be done by using different layers of abstraction. The first reason is because it is very difficult to conclude from the altered bytecode, which is not human-readable, to the new behaviour of the application. The second reason is because the changes in the Java code are interpreted by the decompiler, which might not reflect the exact behaviour of the code or even worse, cannot be translated at all.\newline
These problems are encountered by analysing the different abstraction levels of code as well as different decompilers.
%START TEXT INPUT

%
The classes.dex file is a crucial component regarding the application’s code security because a reverse engineering attempt is considered successful when the targeted source code has been recovered from the bytecode analysis. Hence studying the DEX file format together with the Dalvik opcode structure is tightly related to both designing a powerful obfuscation technique or an efficient bytecode analysis tool.
\cite{kovachevaMaster}
%


gaining information about a program and its implementation
details, process aims at enabling an analyst to understand the concrete
relation between implementation and functionality, optimal output
of such a process would be the original source code of the application, not possible in general\newline
Therefore, it is necessary for such a process to provide on the one hand abstract information about structure and inter-dependencies and on the other hand result in very detailed information like bytecode and mnemonics that allow interpretation of implementation\newline
hoffentlich starting points für investigations\newline
java, e.g. read the program code faster\newline


was ist reengineering? wie funktioniert es? was ist das ziel?\newline
reverse engineering process makes use of a whole range of different analysis
methodologies and tools.\newline
only consider static analysis tools\newline

IN ORDER TO GET FULL OVERVIEW DEX/SMALI/JAVA -> WARUM?\newline

WAS MACHEN DIE TOOLS IM ALLGEMEINEN? WOZU BENUTZEN WIR SIE?\newline

\url{https://mobilesecuritywiki.com/}\newline
\url{https://net.cs.uni-bonn.de/fileadmin/user_upload/plohmann/2012-Schulz-Code_Protection_in_Android.pdf}\newline
main tools\newline
