\subsection{Detect Code Manipulations} \label{subsection:forensics-tools-diff}
%START TEXT INPUT
This is my real text! Rest might be copied or not be checked!
%START TEXT INPUT


vergleich gibts guten einblick was geändert wurde und wie es auf dem gegebenem lvl funtkioniert\newline

vergleich von original und modifizierten code einer apk auf einer code ebene\newline
needed to see differences before and after cracking tool\newline

diff is used\newline

\lstinputlisting[
  float=h,
  breaklines=true,
  captionpos=b,
  frame=single,
  numbers=left,
  language=bash,
  linerange={1-9},
  firstnumber=1,
  caption={Script to compare the original and manipulated \gls{apk} to see the modifications in the different presentations},
  label={codeSnippet:diffScript}
]{data/diffScript.sh}


erklärung command \cite{diffUbuntu}
-N: Treat absent files as empty; Allows the patch create and remove files.\newline
-a: Treat all files as text; Allows the patch update non-text (aka: binary) files.\newline
-u: Set the default 3 lines of unified context; This generates useful time stamps and context.\newline
-r: Recursively compare any subdirectories found; Allows the patch to update subdirectories.\newline
script erklären\newline

RESULT code snippet~\ref{codeSnippet:n1DiffDex}

wo findet man es?\newline
welches level?\newline
vorteil\newline
blabla aus dem internet\newline
