\subsection{Smali Analysis} \label{subsection:tools-baksmali}
\begin{itemize}
    \item test
\end{itemize}
%START TEXT INPUT
This is my real text! Rest might be copied or not be checked!
%START TEXT INPUT

basically jasmin syntax
%
smali, most popular Dalvik bytecode decompilers (used by multiple reverse engineering tools as a base disassembler, amongst which is the also well-known apktool)
\cite{kovachevaMaster}
%
dex == smali, smali better readable but dex to see how easy change
since the translation from java to dex does some optimizations/logik, dex and java do not express the same, but it is how it is in the decompiled code, java is also an abstraction of the actual code, sometimes java also a little confusing since changes happened in dex code and cannot be decompiled to java in a good manner, very messy, it is included for better understanding anyways since humanreadable


stichwort mnemonics, eine seite dex und auf der anderen seite smali, dex bytecode vs smali, Only a few pieces of information are usually not included like the addresses of instructions\newline
unintuitive representation, deswegen smali mit corresponding mnemonics\newline
mnemonics and vice versa is available due to the bijective mapping\newline
correct startaddress and offset can be challenging. There are two major approaches: linear sweep disassembling and recursive traversal disassembling, The linear sweep algorithm is prone to producing wrong mnemonics e.g. when a assembler inlines data so that instructions and data are interleaved. The recursive traversal algorithm is not prone to this but can be attacked by obfuscation techniques like junkbyte insertion as discussed in section 4.4. So for obfuscation, a valuable attack vector on disassembling is to attack the address finding step of these algorithms\newline

https://github.com/JesusFreke/smali

Smali code is the generated by disassembling Dalvik bytecode using baksmali. The result is a human-readable, assambler-like code

The smali [7] program is an \gls{assemblerg}has own \gls{disassemblerg} called "baksmali"\newline
can be used to unpack, modify, and repack Android applications\newline
interesting part for obfuscation and reverse engineering is baksmali. baksmali is similar to dexdump but uses a recursive traversal approach to find instructions\newline
vorteil? -see- So in this approach the next instruction will be expected at the address where the current instruction can jump to, e.g. for the "goto" instruction. This minimizes some problems connected to the linear sweep approach. baksmali is also used by other reverse engineering tools as a basic disassembler\newline

RESULT OUTPUT:
selbe wie dex, jedoch human readable, no big difference, nebeneinanderstellung dex/smali

SCRIPT (LISTING BENUTZEN UND RICHTIGE APP)
\lstinputlisting[
  float=h,
  breaklines=true,
  captionpos=b,
  frame=single,
  numbers=left,
  language=bash,
  linerange={19-21},
  firstnumber=1,
  caption={Script to generate the corresponding smali code for a given \gls{apk}},
  label={codeSnippet:smaliScript}
]{data/extractScript.sh}

\begin{lstlisting}[
  float=h,
  basicstyle=\footnotesize,
  breaklines=true,
  captionpos=b,
  frame=single,
  caption={smali example},
  label={codeSnippet:smaliOutput}
]
.method protected onDestroy()V
    .registers 2

    .prologue
    .line 77
    invoke-super {p0}, Landroid/support/v7/app/AppCompatActivity;->onDestroy()V

    .line 78
    iget-object v0, p0, Lme/neutze/licensetest/MainActivity;->mChecker:Lcom/google/android/vending/licensing/LicenseChecker;

    invoke-virtual {v0}, Lcom/google/android/vending/licensing/LicenseChecker;->onDestroy()V

    .line 79
    return-void
.end method
\end{lstlisting}
EXAMPLE BESCHREIBEN code snippet~\ref{codeSnippet:smaliOutput}

jedes tool:\newline
woher kommt es?\newline
wozu wurde es erfunden?\newline
wer hat es erfunden? quelle\newline
blabla von der seite\newline
wozu benutze ich es?\newline
welches abstrahierungslevel\newline
beispiel\newline
additional features?\newline
WARUM SCHAUEN WIR ES UNS AN?\newline
wo findet man es?\newline
welches level?\newline
vorteil\newline
blabla aus dem internet\newline
