\subsection{Retrieving an APK} \label{subsection:tools-apk}
The analysis is performed on a computer since most tools are available on that platform.
In order to analyse the application, its \gls{apk} has to be extracted from the phone and transfered to the computer.
\newline
Most \gls{apk}s are stored in \textit{/data/app}.
The filesystem of an Android device be accesses using the \textit{adb shell}, which is part of the Android \gls{sdk}.
The user has no read rights on the folder and thus cannot see the content of the folder.
In order to pull the desired application package the exact name is required.
The package manager can be used to list all installed applications and their package names.
The list can be retrieved using \textit{pm list packages -f} in the \textit{adb shell}.
In case the phone is \textit{rooted}, the user can list the content of the server in the \textit{adb shell}.
The format of the application package name is \textit{<namespace>.<appName>} and contains the application as \textit{base.apk}.
The \gls{apk} can be extracted to the current folder using \textit{adb pull /data/app/<namespace>.<appName>/base.apk} on the computer's console.
