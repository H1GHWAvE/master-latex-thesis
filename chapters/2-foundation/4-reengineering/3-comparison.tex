\subsection{Code Comparison} \label{subsection:forensics-tools-diff}
The amount of code of the abstraction levels is huge.
Most of the code base of an abstraction layer stays the same after the attack and the changes are only punctually.
Diff a standard command line tool used to compare two files with little differences in an overall unchanged file.
\newline
By comparing the different code abstractions of the original \gls{apk} with the cracked application, changes done by \gls{luckypatcherg} are identified.
Diff is used in a script to generate the result for different applications and abstraction levels at once (see code snippet~\ref{codeSnippet:diffScript}).
Doing this automatically and using diff saves a lot of time.
\newline
\lstinputlisting[
  float=h,
  breaklines=true,
  captionpos=b,
  frame=single,
  numbers=left,
  language=bash,
  linerange={1-9},
  firstnumber=1,
  caption={Script to compare the original and manipulated \gls{apk} to see the modifications in the different presentations},
  label={codeSnippet:diffScript}
]{data/diffScript.sh}
The result does not only contain the change as original and new code, but also the location where the change happened.
The example diff of a dex file is presented is code snippet~\ref{codeSnippet:n1DiffDex}.
\newline
\lstinputlisting[
  style=diff,
  breakatwhitespace=false,
  breaklines=true,
  captionpos=b,
  frame=single,
  linerange={1-3},
  caption={Diff on Dex level for N1 pattern},
  label={codeSnippet:n1DiffDex}
]{data/n1.diff}
