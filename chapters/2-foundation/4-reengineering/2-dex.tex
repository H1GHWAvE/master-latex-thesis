\subsection{dex Analysis} \label{subsection:tools-dex}
The \textit{classes.dex} contains the application code and has to be modified by the cracking tool to carry out the attack.
It makes it possible to point out which bytecodes have been modified.
This is the reason to analyse the dex bytecode and use it as the first abstraction layer.
\newline
The extraction of the \textit{classes.dex} is done using a simple script shown in code snippet~\ref{codeSnippet:dexScript}.
\newline
\lstinputlisting[
  float=h,
  breaklines=true,
  captionpos=b,
  frame=single,
  numbers=left,
  language=bash,
  linerange={32-35},
  firstnumber=1,
  caption={Script to extract the \gls{dex} bytecode from the \gls{apk}},
  label={codeSnippet:dexScript}
]{data/extractScript.sh}
The \gls{apk} is an archive file and can be unpacked using \textit{unzip}.
The content is unpacked to the destination which is added with the parameter \textit{-d location} as seen in line 3.
The extracted \textit{classes.dex} is still formatted in binary.
Hexdump is used to convert it to a hexadecimal view.
In this view, one character is 4 bit, thus one tuple is one byte and two bytes form a 16 bit opcode.
This presentation allows to identify opcodes and translate them using an opcode table \cite{opcodes}.
\newline
The output contains the line number, the bytecode and the ASCII translation.
Code snippet~\ref{codeSnippet:dexOutput} is an example of the beginning of the \textit{classes.dex} file.
The first 8 byte or 16 hex tuples, \textit{64 65 78 0A 30 33 35 00}, are the \gls{dex} file magic, which identifies the file type.
Translated to ASCII, the result is \textit{dex.035.}.
\begin{lstlisting}[
  float=h,
  basicstyle=\ttfamily\small,
  breaklines=true,
  captionpos=b,
  frame=single,
  caption={Hexadecimal view of classes.dex as classes.txt},
  label={codeSnippet:dexOutput}
]
00000000  64 65 78 0a 30 33 35 00  ae a5 51 7e 06 f7 00 84  |dex.035...Q~....|
00000010  ee 23 5d 3b 4a 61 bb 08  51 a7 c9 02 c1 4e d2 91  |.#];Ja..Q....N..|
00000020  0c fb 21 00 70 00 00 00  78 56 34 12 00 00 00 00  |..!.p...xV4.....|
00000030  00 00 00 00 ac 88 06 00  f4 4e 00 00 70 00 00 00  |.........N..p...|
00000040  ad 09 00 00 40 3c 01 00  0a 0e 00 00 f4 62 01 00  |....@<.......b..|
00000050  3d 27 00 00 6c 0b 02 00  ff 4b 00 00 54 45 03 00  |='..l....K..TE..|
\end{lstlisting}
