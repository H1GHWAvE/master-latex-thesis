\subsubsection{Dex} \label{subsubsection:tools-dex}
%START TEXT INPUT
This is my real text! Rest might be copied or not be checked!
%START TEXT INPUT

%
DVM is register based. Registers are considered 32 bits wide to store values such as integers or floating point numbers. Adjacent register pairs are used to store 64-bit values\newline
dest-then-source ordering for its arguments\newline
there are 218 used valid opcodes in Dalvik bytecode -> QUELLE \newline
\cite{kovachevaMaster} \cite{ehringerDalvik}
%



jedes tool:\newline
woher kommt es?\newline
wozu wurde es erfunden?\newline
wer hat es erfunden? quelle\newline
blabla von der seite\newline
wozu benutze ich es?\newline
welches abstrahierungslevel\newline
beispiel\newline
additional features?\newline
WARUM SCHAUEN WIR ES UNS AN?\newline
wo findet man es?\newline
welches level?\newline
vorteil\newline
blabla aus dem internet\newline
für menschen unintuitiv

hexadeximal value, represent exactly one instruction\newline
unintuitiv, instruction set well defined\newline
ggf bezug zu DALVIK/buildprocess
The Dalvik Virtual Machine (DVM) provides also the ability to call native functions within shared objects out of the Dalvik bytecode. When speaking of reverse engineering an Android application we mostly mean to reverse engineer the bytecode located in the dex file of this application.\newline

\begin{lstlisting}[
  basicstyle=\footnotesize,
  breaklines=true,
  captionpos=b,
  frame=single,
  caption={[Name, Quelle]Quelle},
  label={lst:dalvik}
]
  a6 8e 15 00 bd 8e 15 00  d5 8e 15 00 f0 8e 15 00  |................|
\end{lstlisting}

mein custom script erklären\newline
