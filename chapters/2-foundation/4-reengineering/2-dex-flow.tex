\subsection{dex Analysis} \label{subsection:tools-dex}
%START TEXT INPUT
This is my real text! Rest might be copied or not be checked!
%START TEXT INPUT


%
nur dex weil die apps im moment so vorliegen

aosp-supplied dexdumo to disassemble dex

\cite{andevconDalvikART}
%
%
always attack dex since the protection mechanism is in there (except JNI?)
since apk is zip like decompression tool like 7zip can extract classes.dex from apk file

hexdump to get bytecode


code wie er vorliegt, wenn was geändert wird wird es hier geändert



SCRIPT (LISTING BENUTZEN UND RICHTIGE APP)
\lstinputlisting[
  float=h,
  breaklines=true,
  captionpos=b,
  frame=single,
  numbers=left,
  language=bash,
  linerange={32-35},
  firstnumber=1,
  caption={Script to extract the \gls{dex} byte code from the \gls{apk}},
  label={codeSnippet:dexScript}
]{data/extractScript.sh}
RESULT OUTPUT
dex hexdump, start of the dex file, e.g first 8 byte represent the dex header magic dex.035, version 35\cite{developersDalvik}
LINE | bytecode | ASCII representation
\begin{lstlisting}[
  float=h,
  basicstyle=\ttfamily\small,
  breaklines=true,
  captionpos=b,
  frame=single,
  caption={dex hexdump example},
  label={codeSnippet:dexOutput}
]
00000000  64 65 78 0a 30 33 35 00  ae a5 51 7e 06 f7 00 84  |dex.035...Q~....|
00000010  ee 23 5d 3b 4a 61 bb 08  51 a7 c9 02 c1 4e d2 91  |.#];Ja..Q....N..|
00000020  0c fb 21 00 70 00 00 00  78 56 34 12 00 00 00 00  |..!.p...xV4.....|
00000030  00 00 00 00 ac 88 06 00  f4 4e 00 00 70 00 00 00  |.........N..p...|
00000040  ad 09 00 00 40 3c 01 00  0a 0e 00 00 f4 62 01 00  |....@<.......b..|
00000050  3d 27 00 00 6c 0b 02 00  ff 4b 00 00 54 45 03 00  |='..l....K..TE..|
\end{lstlisting}





jedes tool:\newline
woher kommt es?\newline
wozu wurde es erfunden?\newline
wer hat es erfunden? quelle\newline
blabla von der seite\newline
wozu benutze ich es?\newline
welches abstrahierungslevel\newline
beispiel\newline
additional features?\newline
WARUM SCHAUEN WIR ES UNS AN?\newline
wo findet man es?\newline
welches level?\newline
vorteil\newline
blabla aus dem internet\newline
