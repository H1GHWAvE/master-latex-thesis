\subsection{Prevent Debuggability}\label{subsection:tampering-debuggable}
der debug modus kann dem angreifer informationen/logs über die application geben während diese läuft, aus diesen informationen können erkenntnisse über die funktionsweise geben die für einen angriff/modifikation gewonnen werden können. aus diesen informationen können dann patches für software wie lucky patcher entwickelt werden, da man die anzugreifenden stellen bereits kennt.
kann erzwungen werden indem man das debug flag setzt (wo ist es, wie kann es gesetzt werden)\newline
um dies zu verhindern kann gecheckt werden ob dieses flag forciert wird und gegebenenfalls das laufen der application unterbinden\newline

\lstinputlisting[
  basicstyle=\footnotesize,
  breakatwhitespace=false,
  breaklines=true,
  captionpos=b,
  frame=single,
  numbers=left,
  language=Java,
  linerange={14-22},
  firstnumber=14,
  caption={asd},
  label={Code Snippet: luckycode}
]{data/4-countermeasurements/1-tampering/1-debuggable.java}

 Code SNippet~/ref{Code Snippet: luckycode} zeigt eine funktion die auf den debug modus prüft. Dazu werden zuerst in zeile 15 die appinfo auf das debug flag überprüft. ist dieses vorhanden, ist die variable debuggable true. in diesem fall wird dann die geschlossen\newline
