\subsection{Reengineering} \label{subsection:counter-obfuscation-problem}
%START TEXT INPUT
This is my real text! Rest might be copied or not be checked!
%START TEXT INPUT


now that the environment is enforced and the lvl is modified, the next goal is to prevent pirates from even starting to analyze the applcation

does not help when standard version is implemented, that is why this is working best with customized implementation of LVL
%
Reverse engineering and code protection are processes which are opposing each other, neither classified as good nor bad\newline
"good" developer: malware detection and IP protection\newline
"bad" developer: analysis for attack and analysis resistance

\cite{kovachevaMaster}
%
it is not possible to 100 percent evade reengineering, but adding different methods to hide from plain sight of reengineering tools
%
reengineering cannot be vermiede
best is to apply technqiues to make it as hard a possible
\cite{munteanLicense}
%

if they do not see what the app is doing, they cannot fix it

Application developers are interested in protecting their applications. Protection in this case means that it should be hard to understand
what an application is doing and how its functionalities are implemented.\newline


Reverse engineering of Android applications is much easier than on other architectures -see- high level but simple bytecode language\newline

Obfuscation techniques protect intellectual property of software/license verification\newline

possible code obfuscation methods on the Android platform focus on obfuscating Dalvik bytecode -see- limitations of current reverse engineering tools\newline
