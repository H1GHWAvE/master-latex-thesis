%START TEXT INPUT
antwort des tests n icht mehr ok/no sondern verschlüsselter content



Now that that the functionality of LuckyPatcher is analyzed, it is time to investigate in possible solutions for developers. Counter measurements preventing the cracking app from circumventing the license check mechanism are addressed in four different ways.\newline
The first chapter covers functions to discover preconditions in the environment cracking apps use to discover weaknesses or need to be functional. The second chapter uses the aquired knowledge about LuckyPatcher to modify the code resulting in the patching being unsuccessful. In the third chapter presents methods to prevent the reengineering of the developer\'s application and thus the creation of custom cracks. Further hardware and external measurements are explained in the fourth chapter.\newline
%END TEXT INPUT
general suggestions by google  \url{http://android-developers.blogspot.de/2010/09/securing-android-lvl-applications.html}

countermeasurements can be applied at different levels, when creating the software, when compiling the code to dex and on the dex file itself

%
goal is to

amazon/samsung not much to do since from company that is why the following not simple methods target lvl

patching application code is both most wide-spread and most powerful to interfer app logic
ease is unfortunate for android developers, need better methods to protect vulnerable/precious code from attacks
adding additional layer of security, making it a little ahrder for attackers

CHARAKTERIZED IN DIFFERENT TYPES, tampering protection to detect attack, modifications to enforce additional work in order to crack, methods directly targeting reengineering and additional external features, can be stacked
\cite{munteanLicense}
%
