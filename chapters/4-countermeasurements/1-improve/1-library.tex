\subsection{Unique Implementation} \label{subsection:counter-modifications-library}
Attacking with \gls{luckypatcherg} is often successful because developers do not customize the \gls{lvl} at all.
One reason for not customizing the \gls{lvl} is that protection against casual piracy feels sufficient for the developer.
Casual piracy is the copying of \gls{apk}s from one device to another.
\newline
Another reason is that the developers do not know how to fortify the library and thus they do not want to spend additional effort \cite{developersSecuring}.
\newline
While the idea of the approaches is valid for all license verification libraries, the implementation is only possible for the \gls{lvl}.
It requires access to the source code, which is available for the \gls{lvl}, as opposed to Samsung's and Amazon's libraries.
\newline
\gls{luckypatcherg}'s attack is reliant on successfully applying the patching patterns.
For this reason, actively avoiding these patterns should always be the first step to challenge \gls{luckypatcherg}.
This approach is only possible when using the \gls{lvl}, as it requires access to the source code of the license verification mechanism.
Modifying and improving the \gls{lvl} does not only protect from patterns used by \gls{luckypatcherg}, increasing complexity of the application's bytecode makes it unique and harder to reengineer.
In fact, modifying the lvl is what google strongly suggests. \cite{developersSecuring}
\newline
There are three areas the developer should focus on when modifying the \gls{lvl}  \cite{developersSecuring}.
\begin{enumerate}
\item core licensing library logic
\item entry and exit points of the licensing library
\item invocation and handling of the response
\end{enumerate}
The core logic’s two main classes are the \textit{LicenseChecker} and the \textit{LicenseValidator}.
As seen in section~\ref{section:luckypatcher-patterns}, these two classes are the primary target of \gls{luckypatcherg} and thus should be altered as hard as possible while retaining the original function.
Isomorphic code changes can include:
\begin{itemize}
\item replacing a switch statement with a set of if statements and adding additional code between them (see pattern N1)
\item deriving new values for response codes and checking for these values in the further proceeding (see pattern N6)
\item removing unused code for interfaces, e.g. implementing the policy verification inline in the \textit{LicenseValidator} instead of over an interface (see patterns N2, N4, N5 and N6)
\item moving the \gls{lvl} package into the application (see patterns N2 and N7)
\item implementing functions inline where possible (see patterns N2, N5 and N7)
\item making actions in the decompiled code difficult to trace by moving routines to unrelated code (counter intuitive from traditional software engineering)
\item implementing radical response handling, e.g. killing the application without a hint as soon as an invalid server response is detected (results in bad user experience)
\end{itemize}
These are only examples and creativity is welcome since the resulting implementation should be unique. \cite{developersSecuring}
\newline
The entry and exit points can be attacked by creating a counterfeit version of the \gls{lvl} that implements the same interface.
A unique implementation provides resilience against this attack.
It can be achieved by adding additional arguments to the \textit{LicenseChecker} constructor as well as the \textit{allow()} and the \textit{dontAllow()} methods. \cite{developersSecuring}
\newline
Attackers do not only target the \gls{lvl}, but the handling of the result in the application.
This can be prevented by handling the mechanism in a separate activity.
This prevents from scenarios where the attacker voids methods which would prevent further proceeding, e.g. the
In addition, the license verification can be postponed to a later point in time since attackers are expecting it on the the applications launch. \cite{developersSecuring}
\newline
These modifications are easy to apply and make the implementation unique.
There are unlimited ways to implement them.
It needs to be stated, that they are as easily reengineered and patched, but this makes it hard to attack automatically with an \textit{universal} solution.
A determined attacker, willing to invest time and work in disassembling and reassembling, can eventually find a weakness in the code.
This is the usual race of arms between making things secure and finding new ways to break them.
The gained knowledge can be used to create a custom patch for \gls{luckypatcherg}, cracking the application.
The aim of the developer is to make the work of the attacker as hard as possible to the point where the profit is not worth the time. \cite{developersSecuring}
