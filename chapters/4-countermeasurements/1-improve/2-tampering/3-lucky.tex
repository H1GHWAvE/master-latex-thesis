\subsubsection{Lucky Patcher} \label{subsection:counter-improve-tampering-luckypatcher}
Having Lucky Patcher installed is an indicator that the user is attack license verifications libraries.
The check is not as strict as the one for root and exlucding the user from executing is builds upon a foundation.
It can be extended to detect additional unwanted applications by adding their package name to the check \cite{androidCrackingTools}.
Some custom Android versions already include a library called \textit{AntiPiracySupport} \cite{antipiracy} which is used to remove and blacklist piracy applications.
\newline
\lstinputlisting[
  float=h,
  basicstyle=\footnotesize,
  breakatwhitespace=false,
  breaklines=true,
  captionpos=b,
  frame=single,
  numbers=left,
  language=Java,
  linerange={9-69},
  firstnumber=9,
  caption={Partial Listing},
  label={codeSnippet:tamperingLucky}
]{data/4-countermeasurements/1-tampering/3-lucky.java}
As shown in code snippet~\red{codeSnippet:tamperingLucky}, the check tries to aquire whether the Lucky Patcher package is installed.
In case information is available and thus the application is installed, the check stops the application.
