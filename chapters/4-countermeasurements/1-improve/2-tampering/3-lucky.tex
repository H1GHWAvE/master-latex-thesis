\subsubsection{Lucky Patcher} \label{subsection:counter-improve-tampering-luckypatcher}
Having Lucky Patcher installed, is a strong indicator that the user is pirating applications.
The check can be extended to detect additional unwanted applications by adding their package name to the check \cite{androidCrackingTools}.
The check is more specific as the \textit{root} check, as it only excludes people who have a piracy tool.
\newline
\lstinputlisting[
  float=h,
  basicstyle=\footnotesize,
  breakatwhitespace=false,
  breaklines=true,
  captionpos=b,
  frame=single,
  numbers=left,
  language=Java,
  linerange={9-69},
  firstnumber=9,
  caption={Example code for checking whether \gls{luckypatcherg} is installed on the device},
  label={codeSnippet:tamperingLucky}
]{data/4-countermeasurements/1-tampering/3-lucky.java}
As shown in code snippet~\ref{codeSnippet:tamperingLucky}, the check tries to acquire information whether the Lucky Patcher package is installed.
In case information is available and thus the application is installed, the check stops the application.
Again, it might be useful to communicate up front that the application will not run, if Lucky Patcher is present.
