\subsubsection{Sideload} \label{subsection:counter-improve-tampering-sideload}
As modified \gls{apk}s can be created and installed on the target device or any other, checking for \textit{root} and Lucky Patcher is not enough.
Usually, applications, which include a license verification library, are purchased from the corresponding store.
Installing them from other sources is a sign for piracy.
For this reason, developers should enforce installation from trusted sources to ensure that the application is purchased as well.
Some custom Android versions already include a library called \textit{AntiPiracySupport} \cite{antipiracy}.
It has a similar goal and blacklists and disables pirated applications.
\newline
\lstinputlisting[
  float=h,
  basicstyle=\footnotesize,
  breakatwhitespace=false,
  breaklines=true,
  captionpos=b,
  frame=single,
  numbers=left,
  language=Java,
  linerange={15-40},
  firstnumber=15,
  caption={Example code for checking the origin of the installation},
  label={codeSnippet:tamperingSideload}
]{data/4-countermeasurements/1-tampering/4-sideload.java}
The code snippet~\ref{codeSnippet:tamperingSideload} shows the implementation for the stores in scope for the thesis.
Additional stores can and should be added, in case the developer decides to offer the application in another store.
The application will not work when retrieved from a non listed store.
\newline
This feature should be implemented with caution, since Google notes that this method relies on the \textit{getInstallerPackageName} which is neither documented nor supported and \grqq only working by accident\grqq \cite{developersSecuring}.
