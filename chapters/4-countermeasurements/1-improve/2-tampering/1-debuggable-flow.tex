\subsubsection{Debuggability} \label{subsection:section:counter-improve-tampering-debuggable}
Enabling debugging allows the developer to use aditional features for analysing the app at runtime, like printing out logs \cite{androidDebugging}.
This features are used to gain information about the flow of the app and to reenginer functionality.
From the results of this analysis weak points in the application can be identified and custom patches for cracking tools can be derived from them.
This is the reason for disabling the debug flag in release builds on the application stores, but since the flag for debugging is included inside the \gls{dex} file, it can be enabled by attacks
In order to prevent attackers taking advantage from this possiblity, the developer should check whether this flag is activated und thus the application is tampered.
\newline
\lstinputlisting[
  float=h,
  basicstyle=\footnotesize,
  breakatwhitespace=false,
  breaklines=true,
  captionpos=b,
  frame=single,
  numbers=left,
  language=Java,
  linerange={14-22},
  firstnumber=14,
  caption={Example code for checking for debuggability},
  label={codeSnippet:tamperingDebuggable}
]{data/4-countermeasurements/1-tampering/1-debuggable.java}
Code snippet~\ref{codeSnippet:tamperingDebuggable} is an example for an implementation of this check.
The debug flag can be aquired from the application information as seen in line 15.
In case the debug is set, and thus the application is tampered, the process is killed in line 18.
