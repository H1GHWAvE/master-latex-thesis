\subsubsection{Debuggability} \label{subsection:section:counter-improve-tampering-debuggable}
Enabling debugging allows the developer to use additional features for analysing the application at runtime, like printing out logs \cite{androidDebugging}.
These features are used to gain information about the flow of the application and to reengineer functionality.
With the results of this analysis, weak points are identified and custom patches are developed.
\newline
The debug flag indicates that additional information for debugging can be provided.
It is not set in release builds on the application stores, but it can be activated by changing it in the \textit{classes.dex}.
In order to prevent attackers taking advantage, the developer can check whether this flag is activated and the application is tampered.
\lstinputlisting[
  float=h,
  basicstyle=\footnotesize,
  breakatwhitespace=false,
  breaklines=true,
  captionpos=b,
  frame=single,
  numbers=left,
  language=Java,
  linerange={14-22},
  firstnumber=14,
  caption={Example code for checking for debuggability},
  label={codeSnippet:tamperingDebuggable}
]{data/4-countermeasurements/1-tampering/1-debuggable.java}
Code snippet~\ref{codeSnippet:tamperingDebuggable} is an example for an implementation of this check.
The debug flag can be acquired from the application information as seen in line 15.
In case the debug is set, and thus the application is tampered, the process is killed in line 18.
