\subsubsection{Flow Control} \label{subsection:counter-improve-tampering-flow}
Another approach to ensure that the application is not tampered is to define a flow inside the application.
This is not directly a check but enforces the correct sequential exection of code.
The flow has a known sequene and should include which is likely to be tampered.
Inside this code, important parts of the application can be interwoven, e.g. the interface initialization.
As soon as a cracking tool alters the license library and thus the predefined flow, some code blocks are skipped and thus the essential code inside them is not executed.
\newline
A similar technique would be to have two license verifications inside the application.
One is known to fail while the other one is the actual check.
Since \gls{luckypatcherg} does not know the difference, its automatic mode patches both.
The developer can take adventage of this by inverting the callback for the result of the knownly failing verification.
