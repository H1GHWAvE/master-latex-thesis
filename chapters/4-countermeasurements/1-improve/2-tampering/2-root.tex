\subsubsection{Root} \label{subsection:counter-improve-tampering-root}
Root is the foundation to cracking applications and thus when the device is rooted it possible to alter the application.
The developer can check whether root is available on the device and evtually exclude users accordingly.
When using this feature, the developer has to communicate the users the reasons for this strict policy since there are a lot of users who use root for other reasons than cracking applications.
Google has introduced a similar \gls{api}, called SafetyNet \cite{safetynetPay}, which is said to be used in security critical applications like Android Pay \cite{safetynetGoogle}.
\newline
\lstinputlisting[
  float=h,
  basicstyle=\footnotesize,
  breakatwhitespace=false,
  breaklines=true,
  captionpos=b,
  frame=single,
  numbers=left,
  language=Java,
  linerange={16-37},
  firstnumber=16,
  caption={Example code for checking for root},
  label={codeSnippet:tamperingRoot}
]{data/4-countermeasurements/1-tampering/2-root.java}
Since root is achieved by moving the \textit{su} file to the file system, the application can search in the common locations for it.
In case the search is successful, the execution of the application can be terminated.
