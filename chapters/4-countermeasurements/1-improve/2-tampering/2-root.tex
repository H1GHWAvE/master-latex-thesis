\subsubsection{Root} \label{subsection:counter-improve-tampering-root}
\textit{Root} can be used to alter applications or extract protected data.
The developer can check whether \textit{root} is available on the device and eventually exclude these devices from running the application.
The developer needs to communicate to the users the reasons for this strict policy, since there are a lot of users who use \textit{root} for other reasons than cracking applications.
Code snippet~\ref{codeSnippet:tamperingRoot} is an example implementation.
\newline
\lstinputlisting[
  float=h,
  basicstyle=\footnotesize,
  breakatwhitespace=false,
  breaklines=true,
  captionpos=b,
  frame=single,
  numbers=left,
  language=Java,
  linerange={16-37},
  firstnumber=16,
  caption={Example code for checking for \textit{root}},
  label={codeSnippet:tamperingRoot}
]{data/4-countermeasurements/1-tampering/2-root.java}
Since \textit{root} is achieved by the \textit{su} file in the filesystem, the application can search for its existance in the common locations.
In case the search is successful, the execution of the application can be terminated.
\newline
Google has introduced a similar \gls{api}t, called SafetyNet \cite{safetynetGoogle}.
It is used to check the “health and safety of an Android”\cite{safetynetDev}.
It is said to be used in security critical applications like Android Pay and the reason for excluding rooted devices from the service \cite{safetynetGoogle} \cite{safetynetPay} \cite{safetynetPayx}.
