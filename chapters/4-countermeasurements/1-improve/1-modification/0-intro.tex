Attacking with \gls{luckypatcherg} is often successful because many developers do not customize the \gls{lvl} at all.
One reason for not customizing the \gls{lvl} is that protection against casual piracy is sufficient for the developer.
Casual piracy is the copying of \gls{apk}s from one device to another by uninformed users.
/newline
Another reason is that they do not know where exactly to fortify the library and thus they do not want to spend additional effort \cite{developersSecuring}.
\newline
\newline
This thesis presents two approaches to fortify an application against attacks by \gls{luckypatcherg}.
\newline
The first approach is to actively go against \gls{luckypatcherg}'s patterns by modifying the identified parts of code.
\newline
The second approach is to implement the \gls{lvl} with native code which is not targeted by \gls{luckypatcherg}'s auto patching modes.
\newline
While the idea of the approaches is valid for all license verification libraries, the implementation is only possible for the \gls{lvl}.
It requires access to the source code, which is available for the \gls{lvl}, as opposed to Samsung’s and Amazon’s libraries.
