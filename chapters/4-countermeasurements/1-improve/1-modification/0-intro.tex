Attacking with \gls{luckypatcherg} is often successful since many developers do not customize the \gls{lvl}.
One reason for this is developers using the library only against casual piracy.
It prevents users, who try to copy \gls{apk}s directly from one device to another, from using the application.
Another reason is that they do not know where exactly to fortify the library and thus they do not want to spend additional work. \cite{developersSecuring} \cite{munteanLicense}
This thesis presents two approaches for making attacks on the library less succesful.
\newline
The first approach is to actively go against \gls{luckypatcherg}'s patterns by modifying the identified parts of code.
\newline
The second approach is to implement the \gls{lvl} with native code which cannot be targeted by patterns.
