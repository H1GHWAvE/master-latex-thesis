Attacking with \gls{luckypatcherg} is often successful because many developers do not customize the \gls{lvl} at all.
One reason is the protection against casual piracy.
Using the basic implementation prevents the use after copying of the \gls{apk} from one device to another.
This is only effective against uninformed attackers but sufficient for the developer.
Another reason is that they do not know where exactly to fortify the library and thus they do not want to spend additional effort in gathering information. \cite{developersSecuring} \cite{munteanLicense}
This thesis presents two approaches to fortify an application against attacks by \gls{luckypatcherg}.
\newline
The first approach is to actively go against \gls{luckypatcherg}'s patterns by modifying the identified parts of code.
\newline
The second approach is to implement the \gls{lvl} with native code which cannot be targeted is not targeted by \gls{luckypatcherg}'s auto patching modes.
\newline
While the idea of the approaches is valid for all implementations, these approaches are only implementable when using \gls{lvl} since they requires access to the source code.
