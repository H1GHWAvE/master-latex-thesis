The first action is to fortify the spots identified in section~\ref{section:luckypatcher-patterns} when being attacked by \gls{luckypatcherg}.
The goal is to prevent the success of automatic patching and stop execution in case tampering was detected.
Since additional checks can be voided after analysing the code manually and adding them to the patching procedure, obfuscation is introduced as a tool to make reverse engineering more time consuming.
