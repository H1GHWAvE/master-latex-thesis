%START TEXT INPUT
This is my real text! Rest might be copied or not be checked!
%START TEXT INPUT

applied when compiling

%
(a) at source code and (b) bytecode level, Most existing open-source and commercial tools work on source code level\newline
Java code is architecture-independent giving freedom to design generic code transformations. Lowering the obfuscation level to bytecode requires the algorithms applied to be tuned accordingly to the underlying architecture\newline
\cite{kovachevaMaster}
%

%
a few dex obfuscators exist, with different approaches
proguard or sdex, rename methods, field and calss names -- break down string operations so as to chop hard coded strings or encrypt -- can use dynamic class loading (dexloader classes to impede static analysis)
can add dead code and dummy lopps (minor impact of performance)
can also use goto into other onstructions

\cite{andevconDalvikART}
%


%

layout obfuscation
most programmers name their variables, methods and calss in meaning ful way
are preserved in generation of bytecode for dvm, hence still in dex, can be extracted by attacker, gain information and benefit when reengineering
mangles names and ifentifiers that original meaning is lost while preserving correctness of syntax and semantics
result is bytecode can be interpreted but dissable and decompiule provide meaningless name for identifiers etc, e.g single letters or short combinations, welcome for strings section make it smaller
only complicates but does not stop


\cite{munteanLicense}
%

%
will not protect against autoamted attack, does not alter flow of program
makes more difficult for attackers to write initial attack
removing symbols that would quickly reveal original structure
number of commercial and open-source obfuscators available for Java that will work with Android
\cite{developersSecuring}
%

hilft nicht direkt, aber um reengineering besser zu machen

ERWÄHNEN WO IM PROZESS ANGEWENDET\newline


Obfuscators/Optimizors definition\newline
Obfuscation techniques are used to protect software and the implemented algorithms\newline
designed to make reverse engineering harder and more time consuming, hin und her zwischen obfuscation und reverse engineering techniken\newline
obfuscation techniques must not alter the behavior of programs, often only target specific reverse engineering steps, few general protection schemes, possible slower execution, not topic here, just examples for obfuscation applications

remove dead/debug code\newline
potentially encrypt/obfuscate/hide via reflection\newline
\url{https://youtu.be/6vFcEJ2jgOw?t=243}\newline

definition obfuscation, was macht es, wie funktioniert es, wer hat es erfunden, wie wendet man es an\newline

"hard to reverse engineer" but without changing the behavior of this
application, was heißt hard to reverse\newline

parallele zu disassembler ziehen\newline

Obfuscation cannot prevent reverse engineering but can make it harder and more time consuming. We will discuss which obfuscation and code protection methods are applicable under Android and show limitations of current reverse engineering tools\newline

The following optimizers/obfuscators are common tools. (dadrin dann verbreitung preis etc erklären)
