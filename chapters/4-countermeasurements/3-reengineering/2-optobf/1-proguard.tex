\subsubsection{Proguard} \label{subsubsection:counter-reengineering-optobf-proguard}
%START TEXT INPUT
This is my real text! Rest might be copied or not be checked!
%START TEXT INPUT

\url{https://youtu.be/6vFcEJ2jgOw?t=419}\newline

WER HAT ES HERGESTELLT? WAS IST ES? WAS SIND DIE FEATURES? WIE FUNKTIONIERT ES? WIE WIRD ES IMPLEMENTIERT? WIE SIEHT DAS RESULT AUS (EXAMPLE BILD)\newline

ProGuard is an open source tool which is also integrated in the Android SDK
\url{http://proguard.sourceforge.net/}
\url{http://developer.android.com/sdk/index.html}\newline
was ist proguard? was macht er? -> ProGuard is basically a Java obfuscator but can also be used for Android applications because they are usually written in Java // feature set includes identifier obfuscation for packages, classes, methods, and fields\newline
was kann er noch? -> Besides these protection mechanisms it can also identify and highlight dead code so it can be removed in a second, manual step. Unused classes can be removed automatically by ProGuard.\newline
easy integration -> how\newline

\url{http://developer.android.com/tools/help/proguard.html}\newline
optimizes, shrinks, (barely) obfuscates --> free, reduces size, faster\newline
gutes bild \url{https://youtu.be/TNnccRimhsI?t=1360}\newline
removes unnecessary/unused code\newline
merges identical code blocks\newline
performs optimiztations\newline
removes debug information\newline
renames objects\newline
restructures code\newline
removes linenumbers --> stacktrace annoying\newline
\url{https://youtu.be/6vFcEJ2jgOw?t=470}\newline
-->hacker factor 0\newline
does not really help\newline
googles commentar \url{http://android-developers.blogspot.de/2010/09/proguard-android-and-licensing-server.html}\newline

eine art result bzw zusammenfassung -> Without proper naming of classes and methods it is much harder to reverse engineer an application, because in most cases the identifier enables an analyst to directly guess the purpose of the particular part. The program code itself will not be changed heavily, so the obfuscation by this tool is very limited.
