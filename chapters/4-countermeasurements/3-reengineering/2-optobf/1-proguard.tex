\subsubsection{Proguard} \label{subsubsection:counter-reengineering-optobf-proguard}
%START TEXT INPUT
This is my real text! Rest might be copied or not be checked!
%START TEXT INPUT


%
A Java source code obfuscator. ProGuard performs variable identifiers name scrambling for packages, classes, methods and fields. It shrinks the code size by automatically removing unused classes, detects and highlights dead code, but leaves the developer to remove it manually
\cite{kovachevaMaster}
%

%
open source tool
shrinks, optimizes and obfuscates java .class files
result
- smaller apk files (use rprofits download and less space)
- obfuscated code, especially layout obfuscation, harder to reverse engineer
- small performance increase ddue to optimizations
integrated into android build system, thus easy use
default turned off
minifyEnabled true
proguardFiles getDefaultProguardFile('proguard-android.txt), 'proguard-rules.pro'

additional step in build process, right after java compiler compiled to class files, Proguard performs transformation on files
removes unusaed classes, fields, methods and attributes which got past javac
optimization step methods are inlined, unused parameters removed, classes and methods made private/static/final as possible
obfuscation step name and identifiers mangled, data obfuscation is performed, packages flattened, methods renamed to same name and overloading differentiates them

after proguard is finished dx converts to classes.dex


BILD VORHER NACHER
\cite{munteanLicense}
%



\url{https://youtu.be/6vFcEJ2jgOw?t=419}\newline

WER HAT ES HERGESTELLT? WAS IST ES? WAS SIND DIE FEATURES? WIE FUNKTIONIERT ES? WIE WIRD ES IMPLEMENTIERT? WIE SIEHT DAS RESULT AUS (EXAMPLE BILD)\newline

%
identifier mangling, ProGuard uses a similar approach. It uses minimal
lexical-sorted strings like {a, b, c, ..., aa, ab}, original identifiers give information about interesting parts of a program, Reverse engineering methods can use these information to reduce the amount of program code that has to be manually analyzed -see-  neutralizing these information
in order to prevent this reduction,  remove any meta information about the behavior, meaningless string representation holdin respect to consistence means identifiers for the same object must be replaced by the same string,
advantage of minimizing the memory usage, e development process in step ”a” or step ”b” \newline
string obfuscationa, string must be available at runtime because a user cannot
understand an obfuscated or encrypted message dialog, information is context, other is information itself, e.g. key, url, injective function and deobfuscation stub which constructs original at runtime so no behaviour is changed, does not make understanding harder since only stub is added but reduces usable meta information\newline
\cite{schulzLabCourse}
%


EVALUATION:  dynamic analysis beats this and read directly from memory\newline

ProGuard is an open source tool which is also integrated in the Android SDK
\url{http://proguard.sourceforge.net/}
\url{http://developer.android.com/sdk/index.html}\newline
was ist proguard? was macht er? -see- ProGuard is basically a Java obfuscator but can also be used for Android applications because they are usually written in Java // feature set includes identifier obfuscation for packages, classes, methods, and fields\newline
was kann er noch? -see- Besides these protection mechanisms it can also identify and highlight dead code so it can be removed in a second, manual step. Unused classes can be removed automatically by ProGuard.\newline
easy integration -see- how\newline

\url{http://developer.android.com/tools/help/proguard.html}\newline
optimizes, shrinks, (barely) obfuscates --see- free, reduces size, faster\newline
gutes bild \url{https://youtu.be/TNnccRimhsI?t=1360}\newline
removes unnecessary/unused code\newline
merges identical code blocks\newline
performs optimiztations\newline
removes debug information\newline
renames objects\newline
restructures code\newline
removes linenumbers --see- stacktrace annoying\newline
\url{https://youtu.be/6vFcEJ2jgOw?t=470}\newline
--see-hacker factor 0\newline
does not really help\newline
googles commentar \url{http://android-developers.blogspot.de/2010/09/proguard-android-and-licensing-server.html}\newline

eine art result bzw zusammenfassung -see- Without proper naming of classes and methods it is much harder to reverse engineer an application, because in most cases the identifier enables an analyst to directly guess the purpose of the particular part. The program code itself will not be changed heavily, so the obfuscation by this tool is very limited.
