\subsection{ART} \label{section:counter-external-art}
%START TEXT INPUT
This is my real text! Rest might be copied or not be checked!
%START TEXT INPUT

since dex is more like dangerous executable format and bears significant risks to app developers who do not use countermeasures against it

improve ART, already contains machine code which is hard to analyze and thus also difficult to find patches to apply with luckypatcher



%eval
already on the way, cannot be done from one day on the other, but right now not a protection against luckypatcher, will only be a solution when art code included in apks
but why not now?
%
Evaluation
Why is Android not all ART now? Your applications still compile into Dalvik (DEX) code, Final compilation to ART occurs on the device, during install, Even ART binaries have Dalvik embedded in them, Some methods may be left as DEX, to be interpreted, Dalvik is much easier to debug than ART\newline
\cite{andevconDalvikART}
%

%
zu ART.
dex isnt dead yet, even with art
still buried deep inside those oat files
far easier to reverse engineer embedded dex than do so for oat

art is a far more advanced runtime architecture, brings android closer to ios native level performance
vestiges of dex still remain to haunt performance, dex code is still 32bit
very much still a shifting landscape, internal structures keep on changing, google isnt afraid to break compatibility, llvm integration likely to only increas eand improve
for most users the change is smooth, better performance and power consumption, negligible cost binary size increase, minor limitations on dex obfuscation remain, for optimal performance and obfuscation nothing beats JNI

isn't android all dalvik now?
art is runtime but application compile into dex, art is compiled on device during install, art binaries has dalvik embedded, some methods may be left as dex to be interpreted, dalvik is much easier to debug than art --see- evaluation \newline

When creating odex on art it is directly put into art file

\cite{andevconDalvikART}
%
