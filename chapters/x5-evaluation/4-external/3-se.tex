\subsection{Secure Elements}\label{subsection:evaluation-external-secure}
%START TEXT INPUT
This is my real text! Rest might be copied or not be checked!
%START TEXT INPUT

luckypatcher not able to attack since the logic is put onto a secure external device

nicht für boolean

same as internet service but extra hardare has to be bought
people are lazy and do not want to have extra hardware
when integrated into phone, it will take a long time until each phone supports it, e.g. was not able to use it on Nexus 6P and Nexus 7, Linux did not recognize it but mass storage was enabled, Nexus7 said OTG available but it did not work
many different implementation fragmentate the market, there is not one single solution to focus on and push for market wide accepted solution
solution which are out there have major security flaws, smartcard itself can be attacked

encrypt inside app can be cracked when the source code is known
x = encrypt("Hello") // x = "Hello"
function(x)

function(input):
y = decrypt(input) // y = input
case y == "Hello"


DAP Verification .... normalerweise muss jede Applet, die auf so ein Secure Element/Smartcard etc. kommt mit ner Signatur unterschrieben sein ...
%\url{http://www.win.tue.nl/pinpasjc/docs/Card%20Spec%20v2.1.1%20v0303.pdf}


Waehrend ich Exploits finden konnte, die Dir erw. Zugriff geben, wenn du Applets installieren kannst, u.a.
%\url{https://www.cs.ru.nl/E.Poll/papers/cardis08.pdf}
%\url{http://www.uclouvain.be/crypto/wissec2009/static/13.pdf}​
