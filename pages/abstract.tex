\chapter{\abstractname}
\url{http://users.ece.cmu.edu/~koopman/essays/abstract.html}
Motivation
Problem statement
Results
Approach
Conclusions

Android is the biggest mobile \gls{os}.
This makes it target of software piracy.
Developers cannot protect their \gls{ip} because implemented license verification mechanism are attacked and easily voided by cracking tools.
This thesis analyses the cracking tool \gls{luckypatcherg} and presents the way it works.
The findings are that the attacks are executed by modifying different parts of the code.
Since the response from the license server is always binary, \gls{luckypatcherg} does not have to change the library but attacks the decision points.
The result of the evaluation is always ignored and the code is executed as if valid.
The approach to counter \gls{luckypatcherg} is either an unique implementation of the library, a no longer binary decision or improved environment.
As long as the code can be analysed it can be altered.

\begin{itemize}
  \item priacy problem
  \item android different approaches
  \item luckypatcher attacks
  \item does not change methods, but whether their outcome is included
  \item some patterns can be tricked with mopdification but reverse enineerint to easy
  \item freedom vs walled garden, only able to stop piracy when certificate is checked
  \item a lot to do on android
\end{itemize}
