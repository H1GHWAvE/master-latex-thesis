\section{Motivation} \label{subsection:introduction-motivation}
Licensing is also present in Android.
With a market share of almost 82.8\% in Q2 of 2015 \cite{androidShare}, it is the most used mobile \gls{os}.
According to Google, over 1.4 billion active devices in the last 30 days in September 2015 \cite{androidDevices}.
This giant number of Android devices is powered by Google Play \cite{googlePlay}.
Google's marketplace offers different kinds of digital goods, as applications, music or movies, but also hardware.
In the application section of Google Play, user can chose from over 1.6 million applications for Android \cite{statistaAppStore}.
In 2014, Google's marketplace overtook Apple's Appstore, which had a revenue of over \$10 billion back in 2013, and became the biggest application store on a mobile platform \cite{wiwoValue}.
\newpage
The growth comes with advantages.
Some time ago, developers only considered iOS as a profitable platform and thus most applications were developed for Apple's \gls{os}.
They were ported to Android as an afterthought.
Now, with Android's overwhelming market share, they focus heavily on Android \cite{businessProfit}.
But this also creates a downside.
The expanding market for Android, offering many high quality applications, draws the attention of software pirates.
Crackers bypass license mechanisms of applications and offer them for free or use them to distribute malware.
Cracked applications can redirect cash flows as will be seen in subsection~\ref{subsection:foundation-piracy-developers}, which is a lucrative business model.
\newline
Android developers are aware of the situation \cite{developersPiracy} and express their need to protect their \gls{ip} on platforms like xda-developers \cite{xdaPiracy} or stackoverflow \cite{stackoverflowPiracy}.
Many of the developers have problems with the license verification mechanism and name \gls{luckypatcherg} as one of their biggest problems \cite{stackoverflowLucky}.
\newline
\newline
The scope of this thesis is the analysis of the Android cracking application \gls{luckypatcherg} and the suggestion of countermeasures for developers.
