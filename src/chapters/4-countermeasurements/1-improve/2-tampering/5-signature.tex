\subsubsection{Signature} \label{subsection:counter-improve-tampering-signature}
When modifying an application, it has to be signed once more as described in subsection~\ref{subsection:android-install}.
Since \gls{luckypatcherg} does not have the developer’s key, it has to sign the application using a different key.
\newline
When storing the original key’s hash in the code, it can be compared to the current one.
In case they differ, execution can be interrupted.
\newpage
\lstinputlisting[
  float=h,
  basicstyle=\footnotesize,
  breaklines=true,
  captionpos=b,
  frame=single,
  numbers=left,
  language=Java,
  linerange={51-74},
  firstnumber=51,
  caption={Example code for checking the signature of the application},
  label={codeSnippet:tamperingSignature}
]{data/4-countermeasurements/1-tampering/5-signature.java}
The code for the signature check can be seen in code snippet~\ref{codeSnippet:tamperingSignature}.
The original key’s hash has to be provided (see line 53).
The application signing key’s hash is fetched from the package information (line 57ff).
In case it cannot be retrieved or does not match, the check terminates the application.
\newline
The approach is similar to Google Maps inside an application.
When launching the map, the application sends the SHA1 signature and the API key to the server, which verifies whether the application is allowed to display the map \cite{maps}.
